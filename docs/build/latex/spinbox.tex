%% Generated by Sphinx.
\def\sphinxdocclass{report}
\documentclass[letterpaper,10pt,english]{sphinxmanual}
\ifdefined\pdfpxdimen
   \let\sphinxpxdimen\pdfpxdimen\else\newdimen\sphinxpxdimen
\fi \sphinxpxdimen=.75bp\relax
\ifdefined\pdfimageresolution
    \pdfimageresolution= \numexpr \dimexpr1in\relax/\sphinxpxdimen\relax
\fi
%% let collapsible pdf bookmarks panel have high depth per default
\PassOptionsToPackage{bookmarksdepth=5}{hyperref}

\PassOptionsToPackage{booktabs}{sphinx}
\PassOptionsToPackage{colorrows}{sphinx}

\PassOptionsToPackage{warn}{textcomp}
\usepackage[utf8]{inputenc}
\ifdefined\DeclareUnicodeCharacter
% support both utf8 and utf8x syntaxes
  \ifdefined\DeclareUnicodeCharacterAsOptional
    \def\sphinxDUC#1{\DeclareUnicodeCharacter{"#1}}
  \else
    \let\sphinxDUC\DeclareUnicodeCharacter
  \fi
  \sphinxDUC{00A0}{\nobreakspace}
  \sphinxDUC{2500}{\sphinxunichar{2500}}
  \sphinxDUC{2502}{\sphinxunichar{2502}}
  \sphinxDUC{2514}{\sphinxunichar{2514}}
  \sphinxDUC{251C}{\sphinxunichar{251C}}
  \sphinxDUC{2572}{\textbackslash}
\fi
\usepackage{cmap}
\usepackage[T1]{fontenc}
\usepackage{amsmath,amssymb,amstext}
\usepackage{babel}



\usepackage{tgtermes}
\usepackage{tgheros}
\renewcommand{\ttdefault}{txtt}



\usepackage[Bjarne]{fncychap}
\usepackage{sphinx}

\fvset{fontsize=auto}
\usepackage{geometry}


% Include hyperref last.
\usepackage{hyperref}
% Fix anchor placement for figures with captions.
\usepackage{hypcap}% it must be loaded after hyperref.
% Set up styles of URL: it should be placed after hyperref.
\urlstyle{same}

\addto\captionsenglish{\renewcommand{\contentsname}{Contents:}}

\usepackage{sphinxmessages}
\setcounter{tocdepth}{1}



\title{spinbox}
\date{Aug 13, 2024}
\release{0.1}
\author{Jordan M.\@{} R.\@{} Fox}
\newcommand{\sphinxlogo}{\vbox{}}
\renewcommand{\releasename}{Release}
\makeindex
\begin{document}

\ifdefined\shorthandoff
  \ifnum\catcode`\=\string=\active\shorthandoff{=}\fi
  \ifnum\catcode`\"=\active\shorthandoff{"}\fi
\fi

\pagestyle{empty}
\sphinxmaketitle
\pagestyle{plain}
\sphinxtableofcontents
\pagestyle{normal}
\phantomsection\label{\detokenize{index::doc}}


\sphinxstepscope


\chapter{spinbox}
\label{\detokenize{modules:spinbox}}\label{\detokenize{modules::doc}}
\sphinxstepscope


\section{spinbox package}
\label{\detokenize{spinbox:spinbox-package}}\label{\detokenize{spinbox::doc}}

\subsection{Submodules}
\label{\detokenize{spinbox:submodules}}

\subsection{spinbox.core module}
\label{\detokenize{spinbox:module-spinbox.core}}\label{\detokenize{spinbox:spinbox-core-module}}\index{module@\spxentry{module}!spinbox.core@\spxentry{spinbox.core}}\index{spinbox.core@\spxentry{spinbox.core}!module@\spxentry{module}}\index{CoulombCoupling (class in spinbox.core)@\spxentry{CoulombCoupling}\spxextra{class in spinbox.core}}

\begin{fulllineitems}
\phantomsection\label{\detokenize{spinbox:spinbox.core.CoulombCoupling}}
\pysigstartsignatures
\pysiglinewithargsret{\sphinxbfcode{\sphinxupquote{class\DUrole{w}{ }}}\sphinxcode{\sphinxupquote{spinbox.core.}}\sphinxbfcode{\sphinxupquote{CoulombCoupling}}}{\sphinxparam{\DUrole{n}{n\_particles}}\sphinxparamcomma \sphinxparam{\DUrole{n}{file}\DUrole{o}{=}\DUrole{default_value}{None}}}{}
\pysigstopsignatures
\sphinxAtStartPar
Bases: {\hyperref[\detokenize{spinbox:spinbox.core.Coupling}]{\sphinxcrossref{\sphinxcode{\sphinxupquote{Coupling}}}}}

\sphinxAtStartPar
container class for couplings V\textasciicircum{}coul (i,j)
for i, j = 0 .. n\_particles \sphinxhyphen{} 1
\index{random() (spinbox.core.CoulombCoupling method)@\spxentry{random()}\spxextra{spinbox.core.CoulombCoupling method}}

\begin{fulllineitems}
\phantomsection\label{\detokenize{spinbox:spinbox.core.CoulombCoupling.random}}
\pysigstartsignatures
\pysiglinewithargsret{\sphinxbfcode{\sphinxupquote{random}}}{\sphinxparam{\DUrole{n}{scale}}\sphinxparamcomma \sphinxparam{\DUrole{n}{seed}\DUrole{o}{=}\DUrole{default_value}{0}}}{}
\pysigstopsignatures
\end{fulllineitems}

\index{validate() (spinbox.core.CoulombCoupling method)@\spxentry{validate()}\spxextra{spinbox.core.CoulombCoupling method}}

\begin{fulllineitems}
\phantomsection\label{\detokenize{spinbox:spinbox.core.CoulombCoupling.validate}}
\pysigstartsignatures
\pysiglinewithargsret{\sphinxbfcode{\sphinxupquote{validate}}}{}{}
\pysigstopsignatures
\end{fulllineitems}


\end{fulllineitems}

\index{Coupling (class in spinbox.core)@\spxentry{Coupling}\spxextra{class in spinbox.core}}

\begin{fulllineitems}
\phantomsection\label{\detokenize{spinbox:spinbox.core.Coupling}}
\pysigstartsignatures
\pysiglinewithargsret{\sphinxbfcode{\sphinxupquote{class\DUrole{w}{ }}}\sphinxcode{\sphinxupquote{spinbox.core.}}\sphinxbfcode{\sphinxupquote{Coupling}}}{\sphinxparam{\DUrole{n}{n\_particles}\DUrole{p}{:}\DUrole{w}{ }\DUrole{n}{int}}\sphinxparamcomma \sphinxparam{\DUrole{n}{shape}\DUrole{p}{:}\DUrole{w}{ }\DUrole{n}{tuple\DUrole{p}{{[}}int\DUrole{p}{{]}}}}\sphinxparamcomma \sphinxparam{\DUrole{n}{file}\DUrole{o}{=}\DUrole{default_value}{None}}}{}
\pysigstopsignatures
\sphinxAtStartPar
Bases: \sphinxcode{\sphinxupquote{object}}

\sphinxAtStartPar
Base class for coupling arrays.

\sphinxAtStartPar
Set and get are defined like numpy.ndarray objects.

\sphinxAtStartPar
The simplest example would be something like \(g_{\alpha i}\) for \(\alpha=x,y,z\) and \(i=0 \dots A-1\).

\begin{sphinxVerbatim}[commandchars=\\\{\}]
\PYG{n}{A} \PYG{o}{=} \PYG{l+m+mi}{2}
\PYG{n}{g} \PYG{o}{=} \PYG{n}{Coupling}\PYG{p}{(}\PYG{n}{n\PYGZus{}particles}\PYG{o}{=}\PYG{n}{A}\PYG{p}{,} \PYG{n}{shape}\PYG{o}{=}\PYG{p}{(}\PYG{l+m+mi}{3}\PYG{p}{,}\PYG{n}{A}\PYG{p}{)}\PYG{p}{)}    \PYG{c+c1}{\PYGZsh{}initialize to zeros}
\PYG{n}{g}\PYG{p}{[}\PYG{l+m+mi}{0}\PYG{p}{,}\PYG{l+m+mi}{0}\PYG{p}{]} \PYG{o}{=} \PYG{l+m+mf}{1.0}  \PYG{c+c1}{\PYGZsh{} set an entry by hand }
\end{sphinxVerbatim}
\index{copy() (spinbox.core.Coupling method)@\spxentry{copy()}\spxextra{spinbox.core.Coupling method}}

\begin{fulllineitems}
\phantomsection\label{\detokenize{spinbox:spinbox.core.Coupling.copy}}
\pysigstartsignatures
\pysiglinewithargsret{\sphinxbfcode{\sphinxupquote{copy}}}{}{}
\pysigstopsignatures
\end{fulllineitems}

\index{read() (spinbox.core.Coupling method)@\spxentry{read()}\spxextra{spinbox.core.Coupling method}}

\begin{fulllineitems}
\phantomsection\label{\detokenize{spinbox:spinbox.core.Coupling.read}}
\pysigstartsignatures
\pysiglinewithargsret{\sphinxbfcode{\sphinxupquote{read}}}{\sphinxparam{\DUrole{n}{filename}}}{}
\pysigstopsignatures
\end{fulllineitems}


\end{fulllineitems}

\index{ExactPropagator (class in spinbox.core)@\spxentry{ExactPropagator}\spxextra{class in spinbox.core}}

\begin{fulllineitems}
\phantomsection\label{\detokenize{spinbox:spinbox.core.ExactPropagator}}
\pysigstartsignatures
\pysiglinewithargsret{\sphinxbfcode{\sphinxupquote{class\DUrole{w}{ }}}\sphinxcode{\sphinxupquote{spinbox.core.}}\sphinxbfcode{\sphinxupquote{ExactPropagator}}}{\sphinxparam{\DUrole{n}{n\_particles}}\sphinxparamcomma \sphinxparam{\DUrole{n}{isospin}\DUrole{o}{=}\DUrole{default_value}{True}}}{}
\pysigstopsignatures
\sphinxAtStartPar
Bases: \sphinxcode{\sphinxupquote{object}}

\sphinxAtStartPar
The “exact” propagator.
\begin{equation*}
\begin{split}\exp \left( - \sum_n  g_n \hat{v}_n  \right)\end{split}
\end{equation*}
\sphinxAtStartPar
where \(g_n\) is the entire scalar factor (e.g. \(\frac{\delta\tau}{2} A^{\sigma}_{i \alpha j \beta}\), note the phase convention)
and \(\hat{v}_n\)
is the 2\sphinxhyphen{} or 3\sphinxhyphen{}body interaction operator.

\sphinxAtStartPar
Note, this calculation must be done in the complete many\sphinxhyphen{}body basis; it cannot be restricted to product states.

\sphinxAtStartPar
We use a Pade approximant for the matrix exponential. 
The LS term can be represented using a linear approximation or the factorization procedure described in Stefano’s thesis.
\begin{quote}\begin{description}
\sphinxlineitem{Returns}
\sphinxAtStartPar
The exact propagator.

\sphinxlineitem{Return type}
\sphinxAtStartPar
{\hyperref[\detokenize{spinbox:spinbox.core.HilbertOperator}]{\sphinxcrossref{HilbertOperator}}}

\end{description}\end{quote}
\index{force\_coulomb() (spinbox.core.ExactPropagator method)@\spxentry{force\_coulomb()}\spxextra{spinbox.core.ExactPropagator method}}

\begin{fulllineitems}
\phantomsection\label{\detokenize{spinbox:spinbox.core.ExactPropagator.force_coulomb}}
\pysigstartsignatures
\pysiglinewithargsret{\sphinxbfcode{\sphinxupquote{force\_coulomb}}}{\sphinxparam{\DUrole{n}{coupling}\DUrole{p}{:}\DUrole{w}{ }\DUrole{n}{{\hyperref[\detokenize{spinbox:spinbox.core.CoulombCoupling}]{\sphinxcrossref{CoulombCoupling}}}}}\sphinxparamcomma \sphinxparam{\DUrole{n}{i}\DUrole{p}{:}\DUrole{w}{ }\DUrole{n}{int}}\sphinxparamcomma \sphinxparam{\DUrole{n}{j}\DUrole{p}{:}\DUrole{w}{ }\DUrole{n}{int}}}{{ $\rightarrow$ {\hyperref[\detokenize{spinbox:spinbox.core.HilbertOperator}]{\sphinxcrossref{HilbertOperator}}}}}
\pysigstopsignatures
\end{fulllineitems}

\index{force\_coulomb\_onebody() (spinbox.core.ExactPropagator method)@\spxentry{force\_coulomb\_onebody()}\spxextra{spinbox.core.ExactPropagator method}}

\begin{fulllineitems}
\phantomsection\label{\detokenize{spinbox:spinbox.core.ExactPropagator.force_coulomb_onebody}}
\pysigstartsignatures
\pysiglinewithargsret{\sphinxbfcode{\sphinxupquote{force\_coulomb\_onebody}}}{\sphinxparam{\DUrole{n}{coupling}\DUrole{p}{:}\DUrole{w}{ }\DUrole{n}{complex}}\sphinxparamcomma \sphinxparam{\DUrole{n}{i}\DUrole{p}{:}\DUrole{w}{ }\DUrole{n}{int}}}{{ $\rightarrow$ {\hyperref[\detokenize{spinbox:spinbox.core.HilbertOperator}]{\sphinxcrossref{HilbertOperator}}}}}
\pysigstopsignatures
\sphinxAtStartPar
just the one\sphinxhyphen{}body part of the expanded coulomb propagator
for use along with auxiliary field propagators

\end{fulllineitems}

\index{force\_sigma() (spinbox.core.ExactPropagator method)@\spxentry{force\_sigma()}\spxextra{spinbox.core.ExactPropagator method}}

\begin{fulllineitems}
\phantomsection\label{\detokenize{spinbox:spinbox.core.ExactPropagator.force_sigma}}
\pysigstartsignatures
\pysiglinewithargsret{\sphinxbfcode{\sphinxupquote{force\_sigma}}}{\sphinxparam{\DUrole{n}{coupling}\DUrole{p}{:}\DUrole{w}{ }\DUrole{n}{{\hyperref[\detokenize{spinbox:spinbox.core.SigmaCoupling}]{\sphinxcrossref{SigmaCoupling}}}}}\sphinxparamcomma \sphinxparam{\DUrole{n}{i}\DUrole{p}{:}\DUrole{w}{ }\DUrole{n}{int}}\sphinxparamcomma \sphinxparam{\DUrole{n}{j}\DUrole{p}{:}\DUrole{w}{ }\DUrole{n}{int}}}{{ $\rightarrow$ {\hyperref[\detokenize{spinbox:spinbox.core.HilbertOperator}]{\sphinxcrossref{HilbertOperator}}}}}
\pysigstopsignatures
\end{fulllineitems}

\index{force\_sigma\_3b() (spinbox.core.ExactPropagator method)@\spxentry{force\_sigma\_3b()}\spxextra{spinbox.core.ExactPropagator method}}

\begin{fulllineitems}
\phantomsection\label{\detokenize{spinbox:spinbox.core.ExactPropagator.force_sigma_3b}}
\pysigstartsignatures
\pysiglinewithargsret{\sphinxbfcode{\sphinxupquote{force\_sigma\_3b}}}{\sphinxparam{\DUrole{n}{g}\DUrole{p}{:}\DUrole{w}{ }\DUrole{n}{{\hyperref[\detokenize{spinbox:spinbox.core.ThreeBodyCoupling}]{\sphinxcrossref{ThreeBodyCoupling}}}}}\sphinxparamcomma \sphinxparam{\DUrole{n}{i}\DUrole{p}{:}\DUrole{w}{ }\DUrole{n}{int}}\sphinxparamcomma \sphinxparam{\DUrole{n}{j}\DUrole{p}{:}\DUrole{w}{ }\DUrole{n}{int}}\sphinxparamcomma \sphinxparam{\DUrole{n}{k}\DUrole{p}{:}\DUrole{w}{ }\DUrole{n}{int}}}{{ $\rightarrow$ {\hyperref[\detokenize{spinbox:spinbox.core.HilbertOperator}]{\sphinxcrossref{HilbertOperator}}}}}
\pysigstopsignatures
\end{fulllineitems}

\index{force\_sigmatau() (spinbox.core.ExactPropagator method)@\spxentry{force\_sigmatau()}\spxextra{spinbox.core.ExactPropagator method}}

\begin{fulllineitems}
\phantomsection\label{\detokenize{spinbox:spinbox.core.ExactPropagator.force_sigmatau}}
\pysigstartsignatures
\pysiglinewithargsret{\sphinxbfcode{\sphinxupquote{force\_sigmatau}}}{\sphinxparam{\DUrole{n}{coupling}\DUrole{p}{:}\DUrole{w}{ }\DUrole{n}{{\hyperref[\detokenize{spinbox:spinbox.core.SigmaTauCoupling}]{\sphinxcrossref{SigmaTauCoupling}}}}}\sphinxparamcomma \sphinxparam{\DUrole{n}{i}\DUrole{p}{:}\DUrole{w}{ }\DUrole{n}{int}}\sphinxparamcomma \sphinxparam{\DUrole{n}{j}\DUrole{p}{:}\DUrole{w}{ }\DUrole{n}{int}}}{{ $\rightarrow$ {\hyperref[\detokenize{spinbox:spinbox.core.HilbertOperator}]{\sphinxcrossref{HilbertOperator}}}}}
\pysigstopsignatures
\end{fulllineitems}

\index{force\_tau() (spinbox.core.ExactPropagator method)@\spxentry{force\_tau()}\spxextra{spinbox.core.ExactPropagator method}}

\begin{fulllineitems}
\phantomsection\label{\detokenize{spinbox:spinbox.core.ExactPropagator.force_tau}}
\pysigstartsignatures
\pysiglinewithargsret{\sphinxbfcode{\sphinxupquote{force\_tau}}}{\sphinxparam{\DUrole{n}{coupling}\DUrole{p}{:}\DUrole{w}{ }\DUrole{n}{{\hyperref[\detokenize{spinbox:spinbox.core.TauCoupling}]{\sphinxcrossref{TauCoupling}}}}}\sphinxparamcomma \sphinxparam{\DUrole{n}{i}\DUrole{p}{:}\DUrole{w}{ }\DUrole{n}{int}}\sphinxparamcomma \sphinxparam{\DUrole{n}{j}\DUrole{p}{:}\DUrole{w}{ }\DUrole{n}{int}}}{{ $\rightarrow$ {\hyperref[\detokenize{spinbox:spinbox.core.HilbertOperator}]{\sphinxcrossref{HilbertOperator}}}}}
\pysigstopsignatures
\end{fulllineitems}

\index{propagator\_combined() (spinbox.core.ExactPropagator method)@\spxentry{propagator\_combined()}\spxextra{spinbox.core.ExactPropagator method}}

\begin{fulllineitems}
\phantomsection\label{\detokenize{spinbox:spinbox.core.ExactPropagator.propagator_combined}}
\pysigstartsignatures
\pysiglinewithargsret{\sphinxbfcode{\sphinxupquote{propagator\_combined}}}{\sphinxparam{\DUrole{n}{dt}}\sphinxparamcomma \sphinxparam{\DUrole{n}{potential}}\sphinxparamcomma \sphinxparam{\DUrole{n}{sigma}\DUrole{o}{=}\DUrole{default_value}{False}}\sphinxparamcomma \sphinxparam{\DUrole{n}{sigmatau}\DUrole{o}{=}\DUrole{default_value}{False}}\sphinxparamcomma \sphinxparam{\DUrole{n}{tau}\DUrole{o}{=}\DUrole{default_value}{False}}\sphinxparamcomma \sphinxparam{\DUrole{n}{coulomb}\DUrole{o}{=}\DUrole{default_value}{False}}\sphinxparamcomma \sphinxparam{\DUrole{n}{spinorbit}\DUrole{o}{=}\DUrole{default_value}{False}}\sphinxparamcomma \sphinxparam{\DUrole{n}{sigma\_3b}\DUrole{o}{=}\DUrole{default_value}{False}}}{}
\pysigstopsignatures
\end{fulllineitems}

\index{propagator\_spinorbit\_linear() (spinbox.core.ExactPropagator method)@\spxentry{propagator\_spinorbit\_linear()}\spxextra{spinbox.core.ExactPropagator method}}

\begin{fulllineitems}
\phantomsection\label{\detokenize{spinbox:spinbox.core.ExactPropagator.propagator_spinorbit_linear}}
\pysigstartsignatures
\pysiglinewithargsret{\sphinxbfcode{\sphinxupquote{propagator\_spinorbit\_linear}}}{\sphinxparam{\DUrole{n}{coupling}\DUrole{p}{:}\DUrole{w}{ }\DUrole{n}{{\hyperref[\detokenize{spinbox:spinbox.core.SpinOrbitCoupling}]{\sphinxcrossref{SpinOrbitCoupling}}}}}\sphinxparamcomma \sphinxparam{\DUrole{n}{i}\DUrole{p}{:}\DUrole{w}{ }\DUrole{n}{int}}}{{ $\rightarrow$ {\hyperref[\detokenize{spinbox:spinbox.core.HilbertOperator}]{\sphinxcrossref{HilbertOperator}}}}}
\pysigstopsignatures
\end{fulllineitems}

\index{propagator\_spinorbit\_onebody() (spinbox.core.ExactPropagator method)@\spxentry{propagator\_spinorbit\_onebody()}\spxextra{spinbox.core.ExactPropagator method}}

\begin{fulllineitems}
\phantomsection\label{\detokenize{spinbox:spinbox.core.ExactPropagator.propagator_spinorbit_onebody}}
\pysigstartsignatures
\pysiglinewithargsret{\sphinxbfcode{\sphinxupquote{propagator\_spinorbit\_onebody}}}{\sphinxparam{\DUrole{n}{g}\DUrole{p}{:}\DUrole{w}{ }\DUrole{n}{{\hyperref[\detokenize{spinbox:spinbox.core.SpinOrbitCoupling}]{\sphinxcrossref{SpinOrbitCoupling}}}}}\sphinxparamcomma \sphinxparam{\DUrole{n}{i}\DUrole{p}{:}\DUrole{w}{ }\DUrole{n}{int}}}{{ $\rightarrow$ {\hyperref[\detokenize{spinbox:spinbox.core.HilbertOperator}]{\sphinxcrossref{HilbertOperator}}}}}
\pysigstopsignatures
\end{fulllineitems}

\index{propagator\_spinorbit\_twobody() (spinbox.core.ExactPropagator method)@\spxentry{propagator\_spinorbit\_twobody()}\spxextra{spinbox.core.ExactPropagator method}}

\begin{fulllineitems}
\phantomsection\label{\detokenize{spinbox:spinbox.core.ExactPropagator.propagator_spinorbit_twobody}}
\pysigstartsignatures
\pysiglinewithargsret{\sphinxbfcode{\sphinxupquote{propagator\_spinorbit\_twobody}}}{\sphinxparam{\DUrole{n}{g}\DUrole{p}{:}\DUrole{w}{ }\DUrole{n}{{\hyperref[\detokenize{spinbox:spinbox.core.SpinOrbitCoupling}]{\sphinxcrossref{SpinOrbitCoupling}}}}}\sphinxparamcomma \sphinxparam{\DUrole{n}{i}\DUrole{p}{:}\DUrole{w}{ }\DUrole{n}{int}}\sphinxparamcomma \sphinxparam{\DUrole{n}{j}\DUrole{p}{:}\DUrole{w}{ }\DUrole{n}{int}}}{{ $\rightarrow$ {\hyperref[\detokenize{spinbox:spinbox.core.HilbertOperator}]{\sphinxcrossref{HilbertOperator}}}}}
\pysigstopsignatures
\end{fulllineitems}


\end{fulllineitems}

\index{HilbertOperator (class in spinbox.core)@\spxentry{HilbertOperator}\spxextra{class in spinbox.core}}

\begin{fulllineitems}
\phantomsection\label{\detokenize{spinbox:spinbox.core.HilbertOperator}}
\pysigstartsignatures
\pysiglinewithargsret{\sphinxbfcode{\sphinxupquote{class\DUrole{w}{ }}}\sphinxcode{\sphinxupquote{spinbox.core.}}\sphinxbfcode{\sphinxupquote{HilbertOperator}}}{\sphinxparam{\DUrole{n}{n\_particles}\DUrole{p}{:}\DUrole{w}{ }\DUrole{n}{int}}\sphinxparamcomma \sphinxparam{\DUrole{n}{isospin}\DUrole{o}{=}\DUrole{default_value}{True}}}{}
\pysigstopsignatures
\sphinxAtStartPar
Bases: \sphinxcode{\sphinxupquote{object}}

\sphinxAtStartPar
An operator in the “Hilbert basis.
\index{apply\_onebody\_operator() (spinbox.core.HilbertOperator method)@\spxentry{apply\_onebody\_operator()}\spxextra{spinbox.core.HilbertOperator method}}

\begin{fulllineitems}
\phantomsection\label{\detokenize{spinbox:spinbox.core.HilbertOperator.apply_onebody_operator}}
\pysigstartsignatures
\pysiglinewithargsret{\sphinxbfcode{\sphinxupquote{apply\_onebody\_operator}}}{\sphinxparam{\DUrole{n}{particle\_index}\DUrole{p}{:}\DUrole{w}{ }\DUrole{n}{int}}\sphinxparamcomma \sphinxparam{\DUrole{n}{spin\_matrix}\DUrole{p}{:}\DUrole{w}{ }\DUrole{n}{ndarray}}\sphinxparamcomma \sphinxparam{\DUrole{n}{isospin\_matrix}\DUrole{p}{:}\DUrole{w}{ }\DUrole{n}{ndarray}\DUrole{w}{ }\DUrole{o}{=}\DUrole{w}{ }\DUrole{default_value}{None}}}{}
\pysigstopsignatures
\sphinxAtStartPar
Applies a one\sphinxhyphen{}body / single\sphinxhyphen{}particle operator to the \sphinxcode{\sphinxupquote{HilbertOperator}}.
This accounts for the spin\sphinxhyphen{}isospin kronecker product, if isospin is used.
\begin{equation*}
\begin{split}O' = \sigma_{\alpha i} \tau_{\beta i} O\end{split}
\end{equation*}\begin{quote}\begin{description}
\sphinxlineitem{Parameters}\begin{itemize}
\item {} 
\sphinxAtStartPar
\sphinxstyleliteralstrong{\sphinxupquote{particle\_index}} (\sphinxstyleliteralemphasis{\sphinxupquote{int}}) \textendash{} Index of particle to apply onebody operator to, starting from 0.

\item {} 
\sphinxAtStartPar
\sphinxstyleliteralstrong{\sphinxupquote{spin\_matrix}} (\sphinxstyleliteralemphasis{\sphinxupquote{np.ndarray}}) \textendash{} The spin part of the operator, a 2x2 matrix

\item {} 
\sphinxAtStartPar
\sphinxstyleliteralstrong{\sphinxupquote{isospin\_matrix}} (\sphinxstyleliteralemphasis{\sphinxupquote{numpy.ndarray}}\sphinxstyleliteralemphasis{\sphinxupquote{, }}\sphinxstyleliteralemphasis{\sphinxupquote{optional}}) \textendash{} The isospin part of the operator, a 2x2 matrix, defaults to None

\end{itemize}

\sphinxlineitem{Returns}
\sphinxAtStartPar
A copy of the \sphinxcode{\sphinxupquote{HilbertOperator}} with the one\sphinxhyphen{}body operator applied.

\sphinxlineitem{Return type}
\sphinxAtStartPar
{\hyperref[\detokenize{spinbox:spinbox.core.HilbertOperator}]{\sphinxcrossref{HilbertOperator}}}

\end{description}\end{quote}

\end{fulllineitems}

\index{apply\_sigma() (spinbox.core.HilbertOperator method)@\spxentry{apply\_sigma()}\spxextra{spinbox.core.HilbertOperator method}}

\begin{fulllineitems}
\phantomsection\label{\detokenize{spinbox:spinbox.core.HilbertOperator.apply_sigma}}
\pysigstartsignatures
\pysiglinewithargsret{\sphinxbfcode{\sphinxupquote{apply\_sigma}}}{\sphinxparam{\DUrole{n}{particle\_index}\DUrole{p}{:}\DUrole{w}{ }\DUrole{n}{int}}\sphinxparamcomma \sphinxparam{\DUrole{n}{dimension}\DUrole{p}{:}\DUrole{w}{ }\DUrole{n}{int}}}{{ $\rightarrow$ {\hyperref[\detokenize{spinbox:spinbox.core.HilbertOperator}]{\sphinxcrossref{HilbertOperator}}}}}
\pysigstopsignatures
\sphinxAtStartPar
Applies a one\sphinxhyphen{}body sigma spin operator.
\begin{quote}\begin{description}
\sphinxlineitem{Parameters}\begin{itemize}
\item {} 
\sphinxAtStartPar
\sphinxstyleliteralstrong{\sphinxupquote{particle\_index}} (\sphinxstyleliteralemphasis{\sphinxupquote{int}}) \textendash{} Index of particle, staring from 0.

\item {} 
\sphinxAtStartPar
\sphinxstyleliteralstrong{\sphinxupquote{dimension}} (\sphinxstyleliteralemphasis{\sphinxupquote{int}}) \textendash{} Dimension of sigma operator: 0, 1, 2 = x, y, z

\end{itemize}

\sphinxlineitem{Returns}
\sphinxAtStartPar
The resulting \sphinxcode{\sphinxupquote{HilbertOperator}}.

\sphinxlineitem{Return type}
\sphinxAtStartPar
{\hyperref[\detokenize{spinbox:spinbox.core.HilbertOperator}]{\sphinxcrossref{HilbertOperator}}}

\end{description}\end{quote}

\end{fulllineitems}

\index{apply\_tau() (spinbox.core.HilbertOperator method)@\spxentry{apply\_tau()}\spxextra{spinbox.core.HilbertOperator method}}

\begin{fulllineitems}
\phantomsection\label{\detokenize{spinbox:spinbox.core.HilbertOperator.apply_tau}}
\pysigstartsignatures
\pysiglinewithargsret{\sphinxbfcode{\sphinxupquote{apply\_tau}}}{\sphinxparam{\DUrole{n}{particle\_index}\DUrole{p}{:}\DUrole{w}{ }\DUrole{n}{int}}\sphinxparamcomma \sphinxparam{\DUrole{n}{dimension}\DUrole{p}{:}\DUrole{w}{ }\DUrole{n}{int}}}{}
\pysigstopsignatures
\sphinxAtStartPar
Applies a one\sphinxhyphen{}body tau isospin operator.
\begin{quote}\begin{description}
\sphinxlineitem{Parameters}\begin{itemize}
\item {} 
\sphinxAtStartPar
\sphinxstyleliteralstrong{\sphinxupquote{particle\_index}} (\sphinxstyleliteralemphasis{\sphinxupquote{int}}) \textendash{} Index of particle, staring from 0.

\item {} 
\sphinxAtStartPar
\sphinxstyleliteralstrong{\sphinxupquote{dimension}} (\sphinxstyleliteralemphasis{\sphinxupquote{int}}) \textendash{} Dimension of tau operator: 0, 1, 2 = x, y, z

\end{itemize}

\sphinxlineitem{Returns}
\sphinxAtStartPar
The resulting \sphinxcode{\sphinxupquote{HilbertOperator}}.

\sphinxlineitem{Return type}
\sphinxAtStartPar
{\hyperref[\detokenize{spinbox:spinbox.core.HilbertOperator}]{\sphinxcrossref{HilbertOperator}}}

\end{description}\end{quote}

\end{fulllineitems}

\index{copy() (spinbox.core.HilbertOperator method)@\spxentry{copy()}\spxextra{spinbox.core.HilbertOperator method}}

\begin{fulllineitems}
\phantomsection\label{\detokenize{spinbox:spinbox.core.HilbertOperator.copy}}
\pysigstartsignatures
\pysiglinewithargsret{\sphinxbfcode{\sphinxupquote{copy}}}{}{{ $\rightarrow$ {\hyperref[\detokenize{spinbox:spinbox.core.HilbertOperator}]{\sphinxcrossref{HilbertOperator}}}}}
\pysigstopsignatures
\sphinxAtStartPar
Copies the \sphinxcode{\sphinxupquote{HilbertOperator}}.
\begin{quote}\begin{description}
\sphinxlineitem{Returns}
\sphinxAtStartPar
a new instance of \sphinxcode{\sphinxupquote{HilbertOperator}} with all the same properties as self.

\sphinxlineitem{Return type}
\sphinxAtStartPar
{\hyperref[\detokenize{spinbox:spinbox.core.HilbertOperator}]{\sphinxcrossref{HilbertOperator}}}

\end{description}\end{quote}

\end{fulllineitems}

\index{dagger() (spinbox.core.HilbertOperator method)@\spxentry{dagger()}\spxextra{spinbox.core.HilbertOperator method}}

\begin{fulllineitems}
\phantomsection\label{\detokenize{spinbox:spinbox.core.HilbertOperator.dagger}}
\pysigstartsignatures
\pysiglinewithargsret{\sphinxbfcode{\sphinxupquote{dagger}}}{}{{ $\rightarrow$ {\hyperref[\detokenize{spinbox:spinbox.core.HilbertOperator}]{\sphinxcrossref{HilbertOperator}}}}}
\pysigstopsignatures
\sphinxAtStartPar
Hermitian conjugate.
\begin{quote}\begin{description}
\sphinxlineitem{Returns}
\sphinxAtStartPar
The Hermitian conjugate of the original \sphinxcode{\sphinxupquote{HilbertOperator}}.

\sphinxlineitem{Return type}
\sphinxAtStartPar
{\hyperref[\detokenize{spinbox:spinbox.core.HilbertOperator}]{\sphinxcrossref{HilbertOperator}}}

\end{description}\end{quote}

\end{fulllineitems}

\index{exp() (spinbox.core.HilbertOperator method)@\spxentry{exp()}\spxextra{spinbox.core.HilbertOperator method}}

\begin{fulllineitems}
\phantomsection\label{\detokenize{spinbox:spinbox.core.HilbertOperator.exp}}
\pysigstartsignatures
\pysiglinewithargsret{\sphinxbfcode{\sphinxupquote{exp}}}{}{{ $\rightarrow$ {\hyperref[\detokenize{spinbox:spinbox.core.HilbertOperator}]{\sphinxcrossref{HilbertOperator}}}}}
\pysigstopsignatures
\sphinxAtStartPar
Computes the exponential by Pade approximant.
\begin{quote}\begin{description}
\sphinxlineitem{Returns}
\sphinxAtStartPar
Exponentiated operator.

\sphinxlineitem{Return type}
\sphinxAtStartPar
{\hyperref[\detokenize{spinbox:spinbox.core.HilbertOperator}]{\sphinxcrossref{HilbertOperator}}}

\end{description}\end{quote}

\end{fulllineitems}

\index{multiply\_operator() (spinbox.core.HilbertOperator method)@\spxentry{multiply\_operator()}\spxextra{spinbox.core.HilbertOperator method}}

\begin{fulllineitems}
\phantomsection\label{\detokenize{spinbox:spinbox.core.HilbertOperator.multiply_operator}}
\pysigstartsignatures
\pysiglinewithargsret{\sphinxbfcode{\sphinxupquote{multiply\_operator}}}{\sphinxparam{\DUrole{n}{other}\DUrole{p}{:}\DUrole{w}{ }\DUrole{n}{{\hyperref[\detokenize{spinbox:spinbox.core.HilbertOperator}]{\sphinxcrossref{HilbertOperator}}}}}}{{ $\rightarrow$ {\hyperref[\detokenize{spinbox:spinbox.core.HilbertOperator}]{\sphinxcrossref{HilbertOperator}}}}}
\pysigstopsignatures
\sphinxAtStartPar
Multiply two \sphinxcode{\sphinxupquote{HilbertOperator}} instances together to get a new one.
\begin{quote}\begin{description}
\sphinxlineitem{Parameters}
\sphinxAtStartPar
\sphinxstyleliteralstrong{\sphinxupquote{other}} ({\hyperref[\detokenize{spinbox:spinbox.core.HilbertOperator}]{\sphinxcrossref{\sphinxstyleliteralemphasis{\sphinxupquote{HilbertOperator}}}}}) \textendash{} The other \sphinxcode{\sphinxupquote{HilbertOperator}}

\sphinxlineitem{Returns}
\sphinxAtStartPar
The product of the two.

\sphinxlineitem{Return type}
\sphinxAtStartPar
{\hyperref[\detokenize{spinbox:spinbox.core.HilbertOperator}]{\sphinxcrossref{HilbertOperator}}}

\end{description}\end{quote}

\end{fulllineitems}

\index{multiply\_state() (spinbox.core.HilbertOperator method)@\spxentry{multiply\_state()}\spxextra{spinbox.core.HilbertOperator method}}

\begin{fulllineitems}
\phantomsection\label{\detokenize{spinbox:spinbox.core.HilbertOperator.multiply_state}}
\pysigstartsignatures
\pysiglinewithargsret{\sphinxbfcode{\sphinxupquote{multiply\_state}}}{\sphinxparam{\DUrole{n}{other}\DUrole{p}{:}\DUrole{w}{ }\DUrole{n}{{\hyperref[\detokenize{spinbox:spinbox.core.HilbertState}]{\sphinxcrossref{HilbertState}}}}}}{{ $\rightarrow$ {\hyperref[\detokenize{spinbox:spinbox.core.HilbertState}]{\sphinxcrossref{HilbertState}}}}}
\pysigstopsignatures
\sphinxAtStartPar
Apply the operator to a \sphinxcode{\sphinxupquote{HilbertState}} ket.
\begin{quote}\begin{description}
\sphinxlineitem{Parameters}
\sphinxAtStartPar
\sphinxstyleliteralstrong{\sphinxupquote{other}} ({\hyperref[\detokenize{spinbox:spinbox.core.HilbertState}]{\sphinxcrossref{\sphinxstyleliteralemphasis{\sphinxupquote{HilbertState}}}}}) \textendash{} The state, ketwise.

\sphinxlineitem{Returns}
\sphinxAtStartPar
The new state, ketwise.

\sphinxlineitem{Return type}
\sphinxAtStartPar
{\hyperref[\detokenize{spinbox:spinbox.core.HilbertState}]{\sphinxcrossref{HilbertState}}}

\end{description}\end{quote}

\end{fulllineitems}

\index{scale() (spinbox.core.HilbertOperator method)@\spxentry{scale()}\spxextra{spinbox.core.HilbertOperator method}}

\begin{fulllineitems}
\phantomsection\label{\detokenize{spinbox:spinbox.core.HilbertOperator.scale}}
\pysigstartsignatures
\pysiglinewithargsret{\sphinxbfcode{\sphinxupquote{scale}}}{\sphinxparam{\DUrole{n}{other}\DUrole{p}{:}\DUrole{w}{ }\DUrole{n}{complex}}}{{ $\rightarrow$ {\hyperref[\detokenize{spinbox:spinbox.core.HilbertOperator}]{\sphinxcrossref{HilbertOperator}}}}}
\pysigstopsignatures
\sphinxAtStartPar
Scalar multiplication.
\begin{quote}\begin{description}
\sphinxlineitem{Parameters}
\sphinxAtStartPar
\sphinxstyleliteralstrong{\sphinxupquote{other}} (\sphinxstyleliteralemphasis{\sphinxupquote{complex}}) \textendash{} A scalar.

\sphinxlineitem{Returns}
\sphinxAtStartPar
The resulting scaled operator.

\sphinxlineitem{Return type}
\sphinxAtStartPar
{\hyperref[\detokenize{spinbox:spinbox.core.HilbertOperator}]{\sphinxcrossref{HilbertOperator}}}

\end{description}\end{quote}

\end{fulllineitems}

\index{zero() (spinbox.core.HilbertOperator method)@\spxentry{zero()}\spxextra{spinbox.core.HilbertOperator method}}

\begin{fulllineitems}
\phantomsection\label{\detokenize{spinbox:spinbox.core.HilbertOperator.zero}}
\pysigstartsignatures
\pysiglinewithargsret{\sphinxbfcode{\sphinxupquote{zero}}}{}{{ $\rightarrow$ {\hyperref[\detokenize{spinbox:spinbox.core.HilbertOperator}]{\sphinxcrossref{HilbertOperator}}}}}
\pysigstopsignatures
\sphinxAtStartPar
Multiplies by zero.
\begin{quote}\begin{description}
\sphinxlineitem{Returns}
\sphinxAtStartPar
A copy of the \sphinxcode{\sphinxupquote{HilbertOperator}} with all zero coefficients.

\sphinxlineitem{Return type}
\sphinxAtStartPar
{\hyperref[\detokenize{spinbox:spinbox.core.HilbertOperator}]{\sphinxcrossref{HilbertOperator}}}

\end{description}\end{quote}

\end{fulllineitems}


\end{fulllineitems}

\index{HilbertPropagatorHS (class in spinbox.core)@\spxentry{HilbertPropagatorHS}\spxextra{class in spinbox.core}}

\begin{fulllineitems}
\phantomsection\label{\detokenize{spinbox:spinbox.core.HilbertPropagatorHS}}
\pysigstartsignatures
\pysiglinewithargsret{\sphinxbfcode{\sphinxupquote{class\DUrole{w}{ }}}\sphinxcode{\sphinxupquote{spinbox.core.}}\sphinxbfcode{\sphinxupquote{HilbertPropagatorHS}}}{\sphinxparam{\DUrole{n}{n\_particles}\DUrole{p}{:}\DUrole{w}{ }\DUrole{n}{int}}\sphinxparamcomma \sphinxparam{\DUrole{n}{dt}\DUrole{p}{:}\DUrole{w}{ }\DUrole{n}{float}}\sphinxparamcomma \sphinxparam{\DUrole{n}{isospin}\DUrole{o}{=}\DUrole{default_value}{True}}\sphinxparamcomma \sphinxparam{\DUrole{n}{include\_prefactors}\DUrole{o}{=}\DUrole{default_value}{True}}}{}
\pysigstopsignatures
\sphinxAtStartPar
Bases: {\hyperref[\detokenize{spinbox:spinbox.core.Propagator}]{\sphinxcrossref{\sphinxcode{\sphinxupquote{Propagator}}}}}

\sphinxAtStartPar
The two\sphinxhyphen{}body propagator applied by Hubbard\sphinxhyphen{}Stratonovich
\begin{equation*}
\begin{split}\exp \left[ - \frac{\delta\tau}{2} \sum_{\alpha i \beta j} A_{\alpha i \beta j} \hat{o}_{\alpha i} \hat{o}_{\beta j} \right]\end{split}
\end{equation*}\index{factors\_coulomb() (spinbox.core.HilbertPropagatorHS method)@\spxentry{factors\_coulomb()}\spxextra{spinbox.core.HilbertPropagatorHS method}}

\begin{fulllineitems}
\phantomsection\label{\detokenize{spinbox:spinbox.core.HilbertPropagatorHS.factors_coulomb}}
\pysigstartsignatures
\pysiglinewithargsret{\sphinxbfcode{\sphinxupquote{factors\_coulomb}}}{\sphinxparam{\DUrole{n}{coupling}\DUrole{p}{:}\DUrole{w}{ }\DUrole{n}{{\hyperref[\detokenize{spinbox:spinbox.core.Coupling}]{\sphinxcrossref{Coupling}}}}}\sphinxparamcomma \sphinxparam{\DUrole{n}{aux}\DUrole{p}{:}\DUrole{w}{ }\DUrole{n}{list}}}{{ $\rightarrow$ list\DUrole{p}{{[}}{\hyperref[\detokenize{spinbox:spinbox.core.HilbertOperator}]{\sphinxcrossref{HilbertOperator}}}\DUrole{p}{{]}}}}
\pysigstopsignatures
\sphinxAtStartPar
Creates factors of the Coulomb propagator.
\begin{equation*}
\begin{split}\exp \left[ -\frac{\delta\tau}{2} \frac{v_{ij}}{4} (1+\tau_{iz}+ \tau_{jz} + \tau_{iz}\tau_{jz} ) \right] \end{split}
\end{equation*}
\sphinxAtStartPar
The result is a list of (noncommuting) terms so they may be shuffled.
\begin{quote}\begin{description}
\sphinxlineitem{Parameters}\begin{itemize}
\item {} 
\sphinxAtStartPar
\sphinxstyleliteralstrong{\sphinxupquote{coupling}} ({\hyperref[\detokenize{spinbox:spinbox.core.Coupling}]{\sphinxcrossref{\sphinxstyleliteralemphasis{\sphinxupquote{Coupling}}}}}) \textendash{} force coupling array (e.g. \(v_C(r_{ij})\))

\item {} 
\sphinxAtStartPar
\sphinxstyleliteralstrong{\sphinxupquote{aux}} (\sphinxstyleliteralemphasis{\sphinxupquote{list}}) \textendash{} values of auxiliary field, length equal to the number of pairs

\end{itemize}

\sphinxlineitem{Returns}
\sphinxAtStartPar
The list of propagator terms

\sphinxlineitem{Return type}
\sphinxAtStartPar
list{[}{\hyperref[\detokenize{spinbox:spinbox.core.HilbertOperator}]{\sphinxcrossref{HilbertOperator}}}{]}

\end{description}\end{quote}

\end{fulllineitems}

\index{factors\_sigma() (spinbox.core.HilbertPropagatorHS method)@\spxentry{factors\_sigma()}\spxextra{spinbox.core.HilbertPropagatorHS method}}

\begin{fulllineitems}
\phantomsection\label{\detokenize{spinbox:spinbox.core.HilbertPropagatorHS.factors_sigma}}
\pysigstartsignatures
\pysiglinewithargsret{\sphinxbfcode{\sphinxupquote{factors\_sigma}}}{\sphinxparam{\DUrole{n}{coupling}\DUrole{p}{:}\DUrole{w}{ }\DUrole{n}{{\hyperref[\detokenize{spinbox:spinbox.core.Coupling}]{\sphinxcrossref{Coupling}}}}}\sphinxparamcomma \sphinxparam{\DUrole{n}{aux}\DUrole{p}{:}\DUrole{w}{ }\DUrole{n}{list}}}{{ $\rightarrow$ list\DUrole{p}{{[}}{\hyperref[\detokenize{spinbox:spinbox.core.HilbertOperator}]{\sphinxcrossref{HilbertOperator}}}\DUrole{p}{{]}}}}
\pysigstopsignatures
\sphinxAtStartPar
Creates factors of the \(A^\sigma_{\alpha i \beta j} \sigma_{i \alpha} \sigma_{j \beta}\) propagator. 
The result is a list of (noncommuting) terms so they may be shuffled.
\begin{quote}\begin{description}
\sphinxlineitem{Parameters}\begin{itemize}
\item {} 
\sphinxAtStartPar
\sphinxstyleliteralstrong{\sphinxupquote{coupling}} ({\hyperref[\detokenize{spinbox:spinbox.core.Coupling}]{\sphinxcrossref{\sphinxstyleliteralemphasis{\sphinxupquote{Coupling}}}}}) \textendash{} force coupling array (e.g. \(A^\sigma_{\alpha i \beta j}\))

\item {} 
\sphinxAtStartPar
\sphinxstyleliteralstrong{\sphinxupquote{aux}} (\sphinxstyleliteralemphasis{\sphinxupquote{list}}) \textendash{} values of auxiliary field, length equal to the number of pairs*3*3

\end{itemize}

\sphinxlineitem{Returns}
\sphinxAtStartPar
The list of propagator terms

\sphinxlineitem{Return type}
\sphinxAtStartPar
list{[}{\hyperref[\detokenize{spinbox:spinbox.core.HilbertOperator}]{\sphinxcrossref{HilbertOperator}}}{]}

\end{description}\end{quote}

\end{fulllineitems}

\index{factors\_sigmatau() (spinbox.core.HilbertPropagatorHS method)@\spxentry{factors\_sigmatau()}\spxextra{spinbox.core.HilbertPropagatorHS method}}

\begin{fulllineitems}
\phantomsection\label{\detokenize{spinbox:spinbox.core.HilbertPropagatorHS.factors_sigmatau}}
\pysigstartsignatures
\pysiglinewithargsret{\sphinxbfcode{\sphinxupquote{factors\_sigmatau}}}{\sphinxparam{\DUrole{n}{coupling}\DUrole{p}{:}\DUrole{w}{ }\DUrole{n}{{\hyperref[\detokenize{spinbox:spinbox.core.Coupling}]{\sphinxcrossref{Coupling}}}}}\sphinxparamcomma \sphinxparam{\DUrole{n}{aux}\DUrole{p}{:}\DUrole{w}{ }\DUrole{n}{list}}}{{ $\rightarrow$ list\DUrole{p}{{[}}{\hyperref[\detokenize{spinbox:spinbox.core.HilbertOperator}]{\sphinxcrossref{HilbertOperator}}}\DUrole{p}{{]}}}}
\pysigstopsignatures
\sphinxAtStartPar
Creates factors of the \(A^{\sigma\tau}_{\alpha i \beta j} \sigma_{i \alpha} \sigma_{j \beta}\tau_{i\gamma}\tau_{j\gamma}\) propagator. 
The result is a list of (noncommuting) terms so they may be shuffled.
\begin{quote}\begin{description}
\sphinxlineitem{Parameters}\begin{itemize}
\item {} 
\sphinxAtStartPar
\sphinxstyleliteralstrong{\sphinxupquote{coupling}} ({\hyperref[\detokenize{spinbox:spinbox.core.Coupling}]{\sphinxcrossref{\sphinxstyleliteralemphasis{\sphinxupquote{Coupling}}}}}) \textendash{} force coupling array (e.g. \(A^{\sigma\tau}_{\alpha i \beta j}\))

\item {} 
\sphinxAtStartPar
\sphinxstyleliteralstrong{\sphinxupquote{aux}} (\sphinxstyleliteralemphasis{\sphinxupquote{list}}) \textendash{} values of auxiliary field, length equal to the number of pairs*3*3*3

\end{itemize}

\sphinxlineitem{Returns}
\sphinxAtStartPar
The list of propagator terms

\sphinxlineitem{Return type}
\sphinxAtStartPar
list{[}{\hyperref[\detokenize{spinbox:spinbox.core.HilbertOperator}]{\sphinxcrossref{HilbertOperator}}}{]}

\end{description}\end{quote}

\end{fulllineitems}

\index{factors\_spinorbit() (spinbox.core.HilbertPropagatorHS method)@\spxentry{factors\_spinorbit()}\spxextra{spinbox.core.HilbertPropagatorHS method}}

\begin{fulllineitems}
\phantomsection\label{\detokenize{spinbox:spinbox.core.HilbertPropagatorHS.factors_spinorbit}}
\pysigstartsignatures
\pysiglinewithargsret{\sphinxbfcode{\sphinxupquote{factors\_spinorbit}}}{\sphinxparam{\DUrole{n}{coupling}\DUrole{p}{:}\DUrole{w}{ }\DUrole{n}{{\hyperref[\detokenize{spinbox:spinbox.core.Coupling}]{\sphinxcrossref{Coupling}}}}}\sphinxparamcomma \sphinxparam{\DUrole{n}{aux}\DUrole{p}{:}\DUrole{w}{ }\DUrole{n}{list}}}{{ $\rightarrow$ list\DUrole{p}{{[}}{\hyperref[\detokenize{spinbox:spinbox.core.HilbertOperator}]{\sphinxcrossref{HilbertOperator}}}\DUrole{p}{{]}}}}
\pysigstopsignatures
\sphinxAtStartPar
Creates factors of the spin\sphinxhyphen{}orbit propagator.
\begin{equation*}
\begin{split}\exp \left[ - \frac{\delta\tau}{2} v_{LS}(r_{ij}) \mathbf{L}\cdot\mathbf{S} \right]\end{split}
\end{equation*}
\sphinxAtStartPar
The result is a list of (noncommuting) terms so they may be shuffled.
\begin{quote}\begin{description}
\sphinxlineitem{Parameters}\begin{itemize}
\item {} 
\sphinxAtStartPar
\sphinxstyleliteralstrong{\sphinxupquote{coupling}} ({\hyperref[\detokenize{spinbox:spinbox.core.Coupling}]{\sphinxcrossref{\sphinxstyleliteralemphasis{\sphinxupquote{Coupling}}}}}) \textendash{} force coupling array (e.g. \(g^\text{LS}_{\alpha i}\))

\item {} 
\sphinxAtStartPar
\sphinxstyleliteralstrong{\sphinxupquote{aux}} (\sphinxstyleliteralemphasis{\sphinxupquote{list}}) \textendash{} values of auxiliary field, length equal to the number of pairs

\end{itemize}

\sphinxlineitem{Returns}
\sphinxAtStartPar
The list of propagator terms

\sphinxlineitem{Return type}
\sphinxAtStartPar
list{[}{\hyperref[\detokenize{spinbox:spinbox.core.HilbertOperator}]{\sphinxcrossref{HilbertOperator}}}{]}

\end{description}\end{quote}

\end{fulllineitems}

\index{factors\_tau() (spinbox.core.HilbertPropagatorHS method)@\spxentry{factors\_tau()}\spxextra{spinbox.core.HilbertPropagatorHS method}}

\begin{fulllineitems}
\phantomsection\label{\detokenize{spinbox:spinbox.core.HilbertPropagatorHS.factors_tau}}
\pysigstartsignatures
\pysiglinewithargsret{\sphinxbfcode{\sphinxupquote{factors\_tau}}}{\sphinxparam{\DUrole{n}{coupling}\DUrole{p}{:}\DUrole{w}{ }\DUrole{n}{{\hyperref[\detokenize{spinbox:spinbox.core.Coupling}]{\sphinxcrossref{Coupling}}}}}\sphinxparamcomma \sphinxparam{\DUrole{n}{aux}\DUrole{p}{:}\DUrole{w}{ }\DUrole{n}{list}}}{}
\pysigstopsignatures
\sphinxAtStartPar
Creates factors of the \(\tau_{i\gamma}\tau_{j\gamma}\) propagator. 
The result is a list of (noncommuting) terms so they may be shuffled.
\begin{quote}\begin{description}
\sphinxlineitem{Parameters}\begin{itemize}
\item {} 
\sphinxAtStartPar
\sphinxstyleliteralstrong{\sphinxupquote{coupling}} ({\hyperref[\detokenize{spinbox:spinbox.core.Coupling}]{\sphinxcrossref{\sphinxstyleliteralemphasis{\sphinxupquote{Coupling}}}}}) \textendash{} force coupling array (e.g. \(A^{\tau}_{ij}\))

\item {} 
\sphinxAtStartPar
\sphinxstyleliteralstrong{\sphinxupquote{aux}} (\sphinxstyleliteralemphasis{\sphinxupquote{list}}) \textendash{} values of auxiliary field, length equal to the number of pairs*3

\end{itemize}

\sphinxlineitem{Returns}
\sphinxAtStartPar
The list of propagator terms

\sphinxlineitem{Return type}
\sphinxAtStartPar
list{[}{\hyperref[\detokenize{spinbox:spinbox.core.HilbertOperator}]{\sphinxcrossref{HilbertOperator}}}{]}

\end{description}\end{quote}

\end{fulllineitems}

\index{onebody() (spinbox.core.HilbertPropagatorHS method)@\spxentry{onebody()}\spxextra{spinbox.core.HilbertPropagatorHS method}}

\begin{fulllineitems}
\phantomsection\label{\detokenize{spinbox:spinbox.core.HilbertPropagatorHS.onebody}}
\pysigstartsignatures
\pysiglinewithargsret{\sphinxbfcode{\sphinxupquote{onebody}}}{\sphinxparam{\DUrole{n}{z}\DUrole{p}{:}\DUrole{w}{ }\DUrole{n}{complex}}\sphinxparamcomma \sphinxparam{\DUrole{n}{operator}\DUrole{p}{:}\DUrole{w}{ }\DUrole{n}{{\hyperref[\detokenize{spinbox:spinbox.core.HilbertOperator}]{\sphinxcrossref{HilbertOperator}}}}}}{{ $\rightarrow$ {\hyperref[\detokenize{spinbox:spinbox.core.HilbertOperator}]{\sphinxcrossref{HilbertOperator}}}}}
\pysigstopsignatures
\sphinxAtStartPar
A one\sphinxhyphen{}body propagator
\begin{equation*}
\begin{split}\exp \left[ - z \hat{o} \right]\end{split}
\end{equation*}\begin{quote}\begin{description}
\sphinxlineitem{Parameters}\begin{itemize}
\item {} 
\sphinxAtStartPar
\sphinxstyleliteralstrong{\sphinxupquote{z}} (\sphinxstyleliteralemphasis{\sphinxupquote{complex}}) \textendash{} scalar

\item {} 
\sphinxAtStartPar
\sphinxstyleliteralstrong{\sphinxupquote{operator}} ({\hyperref[\detokenize{spinbox:spinbox.core.HilbertOperator}]{\sphinxcrossref{\sphinxstyleliteralemphasis{\sphinxupquote{HilbertOperator}}}}}) \textendash{} one\sphinxhyphen{}body operator

\end{itemize}

\sphinxlineitem{Returns}
\sphinxAtStartPar
The one\sphinxhyphen{}body propagator

\sphinxlineitem{Return type}
\sphinxAtStartPar
{\hyperref[\detokenize{spinbox:spinbox.core.HilbertOperator}]{\sphinxcrossref{HilbertOperator}}}

\end{description}\end{quote}

\end{fulllineitems}

\index{twobody\_sample() (spinbox.core.HilbertPropagatorHS method)@\spxentry{twobody\_sample()}\spxextra{spinbox.core.HilbertPropagatorHS method}}

\begin{fulllineitems}
\phantomsection\label{\detokenize{spinbox:spinbox.core.HilbertPropagatorHS.twobody_sample}}
\pysigstartsignatures
\pysiglinewithargsret{\sphinxbfcode{\sphinxupquote{twobody\_sample}}}{\sphinxparam{\DUrole{n}{z}\DUrole{p}{:}\DUrole{w}{ }\DUrole{n}{complex}}\sphinxparamcomma \sphinxparam{\DUrole{n}{x}\DUrole{p}{:}\DUrole{w}{ }\DUrole{n}{float}}\sphinxparamcomma \sphinxparam{\DUrole{n}{operator\_i}\DUrole{p}{:}\DUrole{w}{ }\DUrole{n}{{\hyperref[\detokenize{spinbox:spinbox.core.HilbertOperator}]{\sphinxcrossref{HilbertOperator}}}}}\sphinxparamcomma \sphinxparam{\DUrole{n}{operator\_j}\DUrole{p}{:}\DUrole{w}{ }\DUrole{n}{{\hyperref[\detokenize{spinbox:spinbox.core.HilbertOperator}]{\sphinxcrossref{HilbertOperator}}}}}}{{ $\rightarrow$ {\hyperref[\detokenize{spinbox:spinbox.core.HilbertOperator}]{\sphinxcrossref{HilbertOperator}}}}}
\pysigstopsignatures
\sphinxAtStartPar
A sample of the two\sphinxhyphen{}body propagator in the integrand of the Hubbard\sphinxhyphen{}Stratonovich transform.
\begin{equation*}
\begin{split}\exp(z) \exp ({x \sqrt{-z} \hat{\sigma}_{i \alpha} }) \exp({x \sqrt{-z} \hat{\sigma}_{j \beta} } )\end{split}
\end{equation*}\begin{quote}\begin{description}
\sphinxlineitem{Parameters}\begin{itemize}
\item {} 
\sphinxAtStartPar
\sphinxstyleliteralstrong{\sphinxupquote{z}} (\sphinxstyleliteralemphasis{\sphinxupquote{complex}}) \textendash{} scalar

\item {} 
\sphinxAtStartPar
\sphinxstyleliteralstrong{\sphinxupquote{x}} (\sphinxstyleliteralemphasis{\sphinxupquote{float}}) \textendash{} auxiliary field value

\item {} 
\sphinxAtStartPar
\sphinxstyleliteralstrong{\sphinxupquote{operator\_i}} ({\hyperref[\detokenize{spinbox:spinbox.core.HilbertOperator}]{\sphinxcrossref{\sphinxstyleliteralemphasis{\sphinxupquote{HilbertOperator}}}}}) \textendash{} operator on particle i

\item {} 
\sphinxAtStartPar
\sphinxstyleliteralstrong{\sphinxupquote{operator\_j}} ({\hyperref[\detokenize{spinbox:spinbox.core.HilbertOperator}]{\sphinxcrossref{\sphinxstyleliteralemphasis{\sphinxupquote{HilbertOperator}}}}}) \textendash{} operator on particle j

\end{itemize}

\sphinxlineitem{Returns}
\sphinxAtStartPar
One sample of the two\sphinxhyphen{}body propagator

\sphinxlineitem{Return type}
\sphinxAtStartPar
{\hyperref[\detokenize{spinbox:spinbox.core.HilbertOperator}]{\sphinxcrossref{HilbertOperator}}}

\end{description}\end{quote}

\end{fulllineitems}


\end{fulllineitems}

\index{HilbertPropagatorRBM (class in spinbox.core)@\spxentry{HilbertPropagatorRBM}\spxextra{class in spinbox.core}}

\begin{fulllineitems}
\phantomsection\label{\detokenize{spinbox:spinbox.core.HilbertPropagatorRBM}}
\pysigstartsignatures
\pysiglinewithargsret{\sphinxbfcode{\sphinxupquote{class\DUrole{w}{ }}}\sphinxcode{\sphinxupquote{spinbox.core.}}\sphinxbfcode{\sphinxupquote{HilbertPropagatorRBM}}}{\sphinxparam{\DUrole{n}{n\_particles}}\sphinxparamcomma \sphinxparam{\DUrole{n}{dt}\DUrole{p}{:}\DUrole{w}{ }\DUrole{n}{float}}\sphinxparamcomma \sphinxparam{\DUrole{n}{isospin}\DUrole{o}{=}\DUrole{default_value}{True}}\sphinxparamcomma \sphinxparam{\DUrole{n}{include\_prefactors}\DUrole{o}{=}\DUrole{default_value}{True}}}{}
\pysigstopsignatures
\sphinxAtStartPar
Bases: {\hyperref[\detokenize{spinbox:spinbox.core.Propagator}]{\sphinxcrossref{\sphinxcode{\sphinxupquote{Propagator}}}}}

\sphinxAtStartPar
exp( \sphinxhyphen{} i z op\_i op\_j )
\index{factors\_coulomb() (spinbox.core.HilbertPropagatorRBM method)@\spxentry{factors\_coulomb()}\spxextra{spinbox.core.HilbertPropagatorRBM method}}

\begin{fulllineitems}
\phantomsection\label{\detokenize{spinbox:spinbox.core.HilbertPropagatorRBM.factors_coulomb}}
\pysigstartsignatures
\pysiglinewithargsret{\sphinxbfcode{\sphinxupquote{factors\_coulomb}}}{\sphinxparam{\DUrole{n}{coupling}\DUrole{p}{:}\DUrole{w}{ }\DUrole{n}{{\hyperref[\detokenize{spinbox:spinbox.core.CoulombCoupling}]{\sphinxcrossref{CoulombCoupling}}}}}\sphinxparamcomma \sphinxparam{\DUrole{n}{aux}\DUrole{p}{:}\DUrole{w}{ }\DUrole{n}{list}}}{}
\pysigstopsignatures
\end{fulllineitems}

\index{factors\_sigma() (spinbox.core.HilbertPropagatorRBM method)@\spxentry{factors\_sigma()}\spxextra{spinbox.core.HilbertPropagatorRBM method}}

\begin{fulllineitems}
\phantomsection\label{\detokenize{spinbox:spinbox.core.HilbertPropagatorRBM.factors_sigma}}
\pysigstartsignatures
\pysiglinewithargsret{\sphinxbfcode{\sphinxupquote{factors\_sigma}}}{\sphinxparam{\DUrole{n}{coupling}\DUrole{p}{:}\DUrole{w}{ }\DUrole{n}{{\hyperref[\detokenize{spinbox:spinbox.core.SigmaCoupling}]{\sphinxcrossref{SigmaCoupling}}}}}\sphinxparamcomma \sphinxparam{\DUrole{n}{aux}\DUrole{p}{:}\DUrole{w}{ }\DUrole{n}{list}}}{}
\pysigstopsignatures
\end{fulllineitems}

\index{factors\_sigma\_3b() (spinbox.core.HilbertPropagatorRBM method)@\spxentry{factors\_sigma\_3b()}\spxextra{spinbox.core.HilbertPropagatorRBM method}}

\begin{fulllineitems}
\phantomsection\label{\detokenize{spinbox:spinbox.core.HilbertPropagatorRBM.factors_sigma_3b}}
\pysigstartsignatures
\pysiglinewithargsret{\sphinxbfcode{\sphinxupquote{factors\_sigma\_3b}}}{\sphinxparam{\DUrole{n}{coupling}\DUrole{p}{:}\DUrole{w}{ }\DUrole{n}{{\hyperref[\detokenize{spinbox:spinbox.core.ThreeBodyCoupling}]{\sphinxcrossref{ThreeBodyCoupling}}}}}\sphinxparamcomma \sphinxparam{\DUrole{n}{aux}\DUrole{p}{:}\DUrole{w}{ }\DUrole{n}{list}}}{}
\pysigstopsignatures
\end{fulllineitems}

\index{factors\_sigmatau() (spinbox.core.HilbertPropagatorRBM method)@\spxentry{factors\_sigmatau()}\spxextra{spinbox.core.HilbertPropagatorRBM method}}

\begin{fulllineitems}
\phantomsection\label{\detokenize{spinbox:spinbox.core.HilbertPropagatorRBM.factors_sigmatau}}
\pysigstartsignatures
\pysiglinewithargsret{\sphinxbfcode{\sphinxupquote{factors\_sigmatau}}}{\sphinxparam{\DUrole{n}{coupling}\DUrole{p}{:}\DUrole{w}{ }\DUrole{n}{{\hyperref[\detokenize{spinbox:spinbox.core.SigmaTauCoupling}]{\sphinxcrossref{SigmaTauCoupling}}}}}\sphinxparamcomma \sphinxparam{\DUrole{n}{aux}\DUrole{p}{:}\DUrole{w}{ }\DUrole{n}{list}}}{}
\pysigstopsignatures
\end{fulllineitems}

\index{factors\_spinorbit() (spinbox.core.HilbertPropagatorRBM method)@\spxentry{factors\_spinorbit()}\spxextra{spinbox.core.HilbertPropagatorRBM method}}

\begin{fulllineitems}
\phantomsection\label{\detokenize{spinbox:spinbox.core.HilbertPropagatorRBM.factors_spinorbit}}
\pysigstartsignatures
\pysiglinewithargsret{\sphinxbfcode{\sphinxupquote{factors\_spinorbit}}}{\sphinxparam{\DUrole{n}{coupling}\DUrole{p}{:}\DUrole{w}{ }\DUrole{n}{{\hyperref[\detokenize{spinbox:spinbox.core.SpinOrbitCoupling}]{\sphinxcrossref{SpinOrbitCoupling}}}}}\sphinxparamcomma \sphinxparam{\DUrole{n}{aux}\DUrole{p}{:}\DUrole{w}{ }\DUrole{n}{list}}}{}
\pysigstopsignatures
\end{fulllineitems}

\index{factors\_tau() (spinbox.core.HilbertPropagatorRBM method)@\spxentry{factors\_tau()}\spxextra{spinbox.core.HilbertPropagatorRBM method}}

\begin{fulllineitems}
\phantomsection\label{\detokenize{spinbox:spinbox.core.HilbertPropagatorRBM.factors_tau}}
\pysigstartsignatures
\pysiglinewithargsret{\sphinxbfcode{\sphinxupquote{factors\_tau}}}{\sphinxparam{\DUrole{n}{coupling}\DUrole{p}{:}\DUrole{w}{ }\DUrole{n}{{\hyperref[\detokenize{spinbox:spinbox.core.TauCoupling}]{\sphinxcrossref{TauCoupling}}}}}\sphinxparamcomma \sphinxparam{\DUrole{n}{aux}\DUrole{p}{:}\DUrole{w}{ }\DUrole{n}{list}}}{}
\pysigstopsignatures
\end{fulllineitems}

\index{onebody() (spinbox.core.HilbertPropagatorRBM method)@\spxentry{onebody()}\spxextra{spinbox.core.HilbertPropagatorRBM method}}

\begin{fulllineitems}
\phantomsection\label{\detokenize{spinbox:spinbox.core.HilbertPropagatorRBM.onebody}}
\pysigstartsignatures
\pysiglinewithargsret{\sphinxbfcode{\sphinxupquote{onebody}}}{\sphinxparam{\DUrole{n}{z}\DUrole{p}{:}\DUrole{w}{ }\DUrole{n}{complex}}\sphinxparamcomma \sphinxparam{\DUrole{n}{operator}\DUrole{p}{:}\DUrole{w}{ }\DUrole{n}{{\hyperref[\detokenize{spinbox:spinbox.core.HilbertOperator}]{\sphinxcrossref{HilbertOperator}}}}}}{}
\pysigstopsignatures
\sphinxAtStartPar
exp (\sphinxhyphen{} z opi)

\end{fulllineitems}

\index{threebody\_sample() (spinbox.core.HilbertPropagatorRBM method)@\spxentry{threebody\_sample()}\spxextra{spinbox.core.HilbertPropagatorRBM method}}

\begin{fulllineitems}
\phantomsection\label{\detokenize{spinbox:spinbox.core.HilbertPropagatorRBM.threebody_sample}}
\pysigstartsignatures
\pysiglinewithargsret{\sphinxbfcode{\sphinxupquote{threebody\_sample}}}{\sphinxparam{\DUrole{n}{z}\DUrole{p}{:}\DUrole{w}{ }\DUrole{n}{float}}\sphinxparamcomma \sphinxparam{\DUrole{n}{h\_list}\DUrole{p}{:}\DUrole{w}{ }\DUrole{n}{list}}\sphinxparamcomma \sphinxparam{\DUrole{n}{onebody\_matrix\_i}}\sphinxparamcomma \sphinxparam{\DUrole{n}{onebody\_matrix\_j}}\sphinxparamcomma \sphinxparam{\DUrole{n}{onebody\_matrix\_k}}}{}
\pysigstopsignatures
\sphinxAtStartPar
three body RBM sample written for one combined 3\sphinxhyphen{}body RBM kernel function

\end{fulllineitems}

\index{threebody\_sample\_partial() (spinbox.core.HilbertPropagatorRBM method)@\spxentry{threebody\_sample\_partial()}\spxextra{spinbox.core.HilbertPropagatorRBM method}}

\begin{fulllineitems}
\phantomsection\label{\detokenize{spinbox:spinbox.core.HilbertPropagatorRBM.threebody_sample_partial}}
\pysigstartsignatures
\pysiglinewithargsret{\sphinxbfcode{\sphinxupquote{threebody\_sample\_partial}}}{\sphinxparam{\DUrole{n}{z}\DUrole{p}{:}\DUrole{w}{ }\DUrole{n}{float}}\sphinxparamcomma \sphinxparam{\DUrole{n}{h\_list}\DUrole{p}{:}\DUrole{w}{ }\DUrole{n}{list}}\sphinxparamcomma \sphinxparam{\DUrole{n}{operator\_i}\DUrole{p}{:}\DUrole{w}{ }\DUrole{n}{{\hyperref[\detokenize{spinbox:spinbox.core.HilbertOperator}]{\sphinxcrossref{HilbertOperator}}}}}\sphinxparamcomma \sphinxparam{\DUrole{n}{operator\_j}\DUrole{p}{:}\DUrole{w}{ }\DUrole{n}{{\hyperref[\detokenize{spinbox:spinbox.core.HilbertOperator}]{\sphinxcrossref{HilbertOperator}}}}}\sphinxparamcomma \sphinxparam{\DUrole{n}{operator\_k}\DUrole{p}{:}\DUrole{w}{ }\DUrole{n}{{\hyperref[\detokenize{spinbox:spinbox.core.HilbertOperator}]{\sphinxcrossref{HilbertOperator}}}}}}{}
\pysigstopsignatures
\sphinxAtStartPar
three body propagator sample using three 2\sphinxhyphen{}body RBMs

\end{fulllineitems}

\index{twobody\_sample() (spinbox.core.HilbertPropagatorRBM method)@\spxentry{twobody\_sample()}\spxextra{spinbox.core.HilbertPropagatorRBM method}}

\begin{fulllineitems}
\phantomsection\label{\detokenize{spinbox:spinbox.core.HilbertPropagatorRBM.twobody_sample}}
\pysigstartsignatures
\pysiglinewithargsret{\sphinxbfcode{\sphinxupquote{twobody\_sample}}}{\sphinxparam{\DUrole{n}{z}\DUrole{p}{:}\DUrole{w}{ }\DUrole{n}{float}}\sphinxparamcomma \sphinxparam{\DUrole{n}{h}\DUrole{p}{:}\DUrole{w}{ }\DUrole{n}{int}}\sphinxparamcomma \sphinxparam{\DUrole{n}{operator\_i}\DUrole{p}{:}\DUrole{w}{ }\DUrole{n}{{\hyperref[\detokenize{spinbox:spinbox.core.HilbertOperator}]{\sphinxcrossref{HilbertOperator}}}}}\sphinxparamcomma \sphinxparam{\DUrole{n}{operator\_j}\DUrole{p}{:}\DUrole{w}{ }\DUrole{n}{{\hyperref[\detokenize{spinbox:spinbox.core.HilbertOperator}]{\sphinxcrossref{HilbertOperator}}}}}}{}
\pysigstopsignatures
\end{fulllineitems}


\end{fulllineitems}

\index{HilbertState (class in spinbox.core)@\spxentry{HilbertState}\spxextra{class in spinbox.core}}

\begin{fulllineitems}
\phantomsection\label{\detokenize{spinbox:spinbox.core.HilbertState}}
\pysigstartsignatures
\pysiglinewithargsret{\sphinxbfcode{\sphinxupquote{class\DUrole{w}{ }}}\sphinxcode{\sphinxupquote{spinbox.core.}}\sphinxbfcode{\sphinxupquote{HilbertState}}}{\sphinxparam{\DUrole{n}{n\_particles}\DUrole{p}{:}\DUrole{w}{ }\DUrole{n}{int}}\sphinxparamcomma \sphinxparam{\DUrole{n}{coefficients}\DUrole{o}{=}\DUrole{default_value}{None}}\sphinxparamcomma \sphinxparam{\DUrole{n}{ketwise}\DUrole{o}{=}\DUrole{default_value}{True}}\sphinxparamcomma \sphinxparam{\DUrole{n}{isospin}\DUrole{o}{=}\DUrole{default_value}{True}}}{}
\pysigstopsignatures
\sphinxAtStartPar
Bases: \sphinxcode{\sphinxupquote{object}}

\sphinxAtStartPar
A spin state in the “Hilbert” basis, a linear combination of tensor product states.

\sphinxAtStartPar
States must be defined with a number of particles. 
If \sphinxcode{\sphinxupquote{isospin}} is False, then the one\sphinxhyphen{}body basis is only spin up/down. If True, then it is (spin up/down x isospin up/down).
\sphinxcode{\sphinxupquote{ketwise}} detemines if it is a bra or a ket.
\index{attach\_coordinates() (spinbox.core.HilbertState method)@\spxentry{attach\_coordinates()}\spxextra{spinbox.core.HilbertState method}}

\begin{fulllineitems}
\phantomsection\label{\detokenize{spinbox:spinbox.core.HilbertState.attach_coordinates}}
\pysigstartsignatures
\pysiglinewithargsret{\sphinxbfcode{\sphinxupquote{attach\_coordinates}}}{\sphinxparam{\DUrole{n}{coordinates}\DUrole{p}{:}\DUrole{w}{ }\DUrole{n}{ndarray}}}{}
\pysigstopsignatures
\sphinxAtStartPar
Adds a new \sphinxcode{\sphinxupquote{.coordinates}} attribute to the \sphinxcode{\sphinxupquote{HilbertState}}
\begin{quote}\begin{description}
\sphinxlineitem{Parameters}
\sphinxAtStartPar
\sphinxstyleliteralstrong{\sphinxupquote{coordinates}} (\sphinxstyleliteralemphasis{\sphinxupquote{np.ndarray}}) \textendash{} A Numpy array with shape \sphinxcode{\sphinxupquote{(n\_particles , 3)}} (e.g. x, y, z)

\end{description}\end{quote}

\end{fulllineitems}

\index{copy() (spinbox.core.HilbertState method)@\spxentry{copy()}\spxextra{spinbox.core.HilbertState method}}

\begin{fulllineitems}
\phantomsection\label{\detokenize{spinbox:spinbox.core.HilbertState.copy}}
\pysigstartsignatures
\pysiglinewithargsret{\sphinxbfcode{\sphinxupquote{copy}}}{}{}
\pysigstopsignatures
\sphinxAtStartPar
Copies the \sphinxcode{\sphinxupquote{HilbertState}}.
\begin{quote}\begin{description}
\sphinxlineitem{Returns}
\sphinxAtStartPar
a new instance of \sphinxcode{\sphinxupquote{HilbertState}} with all the same properties.

\sphinxlineitem{Return type}
\sphinxAtStartPar
{\hyperref[\detokenize{spinbox:spinbox.core.HilbertState}]{\sphinxcrossref{HilbertState}}}

\end{description}\end{quote}

\end{fulllineitems}

\index{dagger() (spinbox.core.HilbertState method)@\spxentry{dagger()}\spxextra{spinbox.core.HilbertState method}}

\begin{fulllineitems}
\phantomsection\label{\detokenize{spinbox:spinbox.core.HilbertState.dagger}}
\pysigstartsignatures
\pysiglinewithargsret{\sphinxbfcode{\sphinxupquote{dagger}}}{}{{ $\rightarrow$ {\hyperref[\detokenize{spinbox:spinbox.core.HilbertState}]{\sphinxcrossref{HilbertState}}}}}
\pysigstopsignatures
\sphinxAtStartPar
Hermitian conjugate.
\begin{quote}\begin{description}
\sphinxlineitem{Returns}
\sphinxAtStartPar
The dual \sphinxcode{\sphinxupquote{HilbertState}}

\sphinxlineitem{Return type}
\sphinxAtStartPar
{\hyperref[\detokenize{spinbox:spinbox.core.HilbertState}]{\sphinxcrossref{HilbertState}}}

\end{description}\end{quote}

\end{fulllineitems}

\index{entropy() (spinbox.core.HilbertState method)@\spxentry{entropy()}\spxextra{spinbox.core.HilbertState method}}

\begin{fulllineitems}
\phantomsection\label{\detokenize{spinbox:spinbox.core.HilbertState.entropy}}
\pysigstartsignatures
\pysiglinewithargsret{\sphinxbfcode{\sphinxupquote{entropy}}}{}{{ $\rightarrow$ complex}}
\pysigstopsignatures
\sphinxAtStartPar
Von Neumann entropy, a measure of entanglement.
\begin{quote}\begin{description}
\sphinxlineitem{Returns}
\sphinxAtStartPar
VN entropy of the \sphinxcode{\sphinxupquote{HilbertState}}

\sphinxlineitem{Return type}
\sphinxAtStartPar
complex

\end{description}\end{quote}

\end{fulllineitems}

\index{generate\_basis\_states() (spinbox.core.HilbertState method)@\spxentry{generate\_basis\_states()}\spxextra{spinbox.core.HilbertState method}}

\begin{fulllineitems}
\phantomsection\label{\detokenize{spinbox:spinbox.core.HilbertState.generate_basis_states}}
\pysigstartsignatures
\pysiglinewithargsret{\sphinxbfcode{\sphinxupquote{generate\_basis\_states}}}{}{{ $\rightarrow$ list}}
\pysigstopsignatures
\sphinxAtStartPar
Makes a list of corresponding basis vectors.
\begin{quote}\begin{description}
\sphinxlineitem{Returns}
\sphinxAtStartPar
A list of tensor product states that span the Hilbert space.

\sphinxlineitem{Return type}
\sphinxAtStartPar
list

\end{description}\end{quote}

\end{fulllineitems}

\index{inner() (spinbox.core.HilbertState method)@\spxentry{inner()}\spxextra{spinbox.core.HilbertState method}}

\begin{fulllineitems}
\phantomsection\label{\detokenize{spinbox:spinbox.core.HilbertState.inner}}
\pysigstartsignatures
\pysiglinewithargsret{\sphinxbfcode{\sphinxupquote{inner}}}{\sphinxparam{\DUrole{n}{other}\DUrole{p}{:}\DUrole{w}{ }\DUrole{n}{{\hyperref[\detokenize{spinbox:spinbox.core.HilbertState}]{\sphinxcrossref{HilbertState}}}}}}{{ $\rightarrow$ complex}}
\pysigstopsignatures
\sphinxAtStartPar
Inner product of two HilbertState instances. Orientations must be correct.
\begin{quote}\begin{description}
\sphinxlineitem{Parameters}
\sphinxAtStartPar
\sphinxstyleliteralstrong{\sphinxupquote{other}} ({\hyperref[\detokenize{spinbox:spinbox.core.HilbertState}]{\sphinxcrossref{\sphinxstyleliteralemphasis{\sphinxupquote{HilbertState}}}}}) \textendash{} The ket of the inner product.

\sphinxlineitem{Returns}
\sphinxAtStartPar
inner product of self (bra) with other (ket)

\sphinxlineitem{Return type}
\sphinxAtStartPar
complex

\end{description}\end{quote}

\end{fulllineitems}

\index{multiply\_operator() (spinbox.core.HilbertState method)@\spxentry{multiply\_operator()}\spxextra{spinbox.core.HilbertState method}}

\begin{fulllineitems}
\phantomsection\label{\detokenize{spinbox:spinbox.core.HilbertState.multiply_operator}}
\pysigstartsignatures
\pysiglinewithargsret{\sphinxbfcode{\sphinxupquote{multiply\_operator}}}{\sphinxparam{\DUrole{n}{other}\DUrole{p}{:}\DUrole{w}{ }\DUrole{n}{{\hyperref[\detokenize{spinbox:spinbox.core.HilbertOperator}]{\sphinxcrossref{HilbertOperator}}}}}}{{ $\rightarrow$ {\hyperref[\detokenize{spinbox:spinbox.core.HilbertState}]{\sphinxcrossref{HilbertState}}}}}
\pysigstopsignatures
\sphinxAtStartPar
Multiplies a (bra) \sphinxcode{\sphinxupquote{HilbertState}} on a \sphinxcode{\sphinxupquote{HilbertOperator}}.
\begin{quote}\begin{description}
\sphinxlineitem{Parameters}
\sphinxAtStartPar
\sphinxstyleliteralstrong{\sphinxupquote{other}} ({\hyperref[\detokenize{spinbox:spinbox.core.HilbertOperator}]{\sphinxcrossref{\sphinxstyleliteralemphasis{\sphinxupquote{HilbertOperator}}}}}) \textendash{} The operator.

\sphinxlineitem{Returns}
\sphinxAtStartPar
\textless{} self| O(other)

\sphinxlineitem{Return type}
\sphinxAtStartPar
{\hyperref[\detokenize{spinbox:spinbox.core.HilbertState}]{\sphinxcrossref{HilbertState}}}

\end{description}\end{quote}

\end{fulllineitems}

\index{nearby\_product\_state() (spinbox.core.HilbertState method)@\spxentry{nearby\_product\_state()}\spxextra{spinbox.core.HilbertState method}}

\begin{fulllineitems}
\phantomsection\label{\detokenize{spinbox:spinbox.core.HilbertState.nearby_product_state}}
\pysigstartsignatures
\pysiglinewithargsret{\sphinxbfcode{\sphinxupquote{nearby\_product\_state}}}{\sphinxparam{\DUrole{n}{seed}\DUrole{p}{:}\DUrole{w}{ }\DUrole{n}{int}\DUrole{w}{ }\DUrole{o}{=}\DUrole{w}{ }\DUrole{default_value}{None}}\sphinxparamcomma \sphinxparam{\DUrole{n}{maxiter}\DUrole{o}{=}\DUrole{default_value}{100}}}{}
\pysigstopsignatures
\sphinxAtStartPar
Finds a \sphinxcode{\sphinxupquote{ProductState}} that has a large overlap with the \sphinxcode{\sphinxupquote{HilbertState}}.
\begin{quote}\begin{description}
\sphinxlineitem{Parameters}\begin{itemize}
\item {} 
\sphinxAtStartPar
\sphinxstyleliteralstrong{\sphinxupquote{seed}} (\sphinxstyleliteralemphasis{\sphinxupquote{int}}\sphinxstyleliteralemphasis{\sphinxupquote{, }}\sphinxstyleliteralemphasis{\sphinxupquote{optional}}) \textendash{} RNG seed, defaults to None

\item {} 
\sphinxAtStartPar
\sphinxstyleliteralstrong{\sphinxupquote{maxiter}} (\sphinxstyleliteralemphasis{\sphinxupquote{int}}\sphinxstyleliteralemphasis{\sphinxupquote{, }}\sphinxstyleliteralemphasis{\sphinxupquote{optional}}) \textendash{} maximum iterations to do in optimization, defaults to 100

\end{itemize}

\sphinxlineitem{Returns}
\sphinxAtStartPar
a tuple: (fitted \sphinxcode{\sphinxupquote{ProductState}}, optimization result)

\sphinxlineitem{Return type}
\sphinxAtStartPar
({\hyperref[\detokenize{spinbox:spinbox.core.ProductState}]{\sphinxcrossref{ProductState}}}, scipy.OptimizeResult)

\end{description}\end{quote}

\end{fulllineitems}

\index{nearest\_product\_state() (spinbox.core.HilbertState method)@\spxentry{nearest\_product\_state()}\spxextra{spinbox.core.HilbertState method}}

\begin{fulllineitems}
\phantomsection\label{\detokenize{spinbox:spinbox.core.HilbertState.nearest_product_state}}
\pysigstartsignatures
\pysiglinewithargsret{\sphinxbfcode{\sphinxupquote{nearest\_product\_state}}}{\sphinxparam{\DUrole{n}{seeds}\DUrole{p}{:}\DUrole{w}{ }\DUrole{n}{list\DUrole{p}{{[}}int\DUrole{p}{{]}}}}\sphinxparamcomma \sphinxparam{\DUrole{n}{maxiter}\DUrole{o}{=}\DUrole{default_value}{100}}}{}
\pysigstopsignatures
\sphinxAtStartPar
Does \sphinxcode{\sphinxupquote{self.nearby\_product\_state}} for a list of seeds and returns the result maximizing overlap
\begin{quote}\begin{description}
\sphinxlineitem{Parameters}\begin{itemize}
\item {} 
\sphinxAtStartPar
\sphinxstyleliteralstrong{\sphinxupquote{seed}} (\sphinxstyleliteralemphasis{\sphinxupquote{int}}\sphinxstyleliteralemphasis{\sphinxupquote{, }}\sphinxstyleliteralemphasis{\sphinxupquote{optional}}) \textendash{} RNG seed, defaults to None

\item {} 
\sphinxAtStartPar
\sphinxstyleliteralstrong{\sphinxupquote{maxiter}} (\sphinxstyleliteralemphasis{\sphinxupquote{int}}\sphinxstyleliteralemphasis{\sphinxupquote{, }}\sphinxstyleliteralemphasis{\sphinxupquote{optional}}) \textendash{} maximum iterations to do in optimization, defaults to 100

\end{itemize}

\sphinxlineitem{Returns}
\sphinxAtStartPar
fitted \sphinxcode{\sphinxupquote{ProductState}}

\sphinxlineitem{Return type}
\sphinxAtStartPar
{\hyperref[\detokenize{spinbox:spinbox.core.ProductState}]{\sphinxcrossref{ProductState}}}

\end{description}\end{quote}

\end{fulllineitems}

\index{outer() (spinbox.core.HilbertState method)@\spxentry{outer()}\spxextra{spinbox.core.HilbertState method}}

\begin{fulllineitems}
\phantomsection\label{\detokenize{spinbox:spinbox.core.HilbertState.outer}}
\pysigstartsignatures
\pysiglinewithargsret{\sphinxbfcode{\sphinxupquote{outer}}}{\sphinxparam{\DUrole{n}{other}\DUrole{p}{:}\DUrole{w}{ }\DUrole{n}{{\hyperref[\detokenize{spinbox:spinbox.core.HilbertState}]{\sphinxcrossref{HilbertState}}}}}}{{ $\rightarrow$ {\hyperref[\detokenize{spinbox:spinbox.core.HilbertOperator}]{\sphinxcrossref{HilbertOperator}}}}}
\pysigstopsignatures
\sphinxAtStartPar
Outer product of two HilbertState instances, producting a HilbertOperator instance. Orientations must be correct.
\begin{quote}\begin{description}
\sphinxlineitem{Parameters}
\sphinxAtStartPar
\sphinxstyleliteralstrong{\sphinxupquote{other}} ({\hyperref[\detokenize{spinbox:spinbox.core.HilbertState}]{\sphinxcrossref{\sphinxstyleliteralemphasis{\sphinxupquote{HilbertState}}}}}) \textendash{} bra part of the outer product

\sphinxlineitem{Returns}
\sphinxAtStartPar
Outer product of self (ket) with other (bra)

\sphinxlineitem{Return type}
\sphinxAtStartPar
{\hyperref[\detokenize{spinbox:spinbox.core.HilbertOperator}]{\sphinxcrossref{HilbertOperator}}}

\end{description}\end{quote}

\end{fulllineitems}

\index{randomize() (spinbox.core.HilbertState method)@\spxentry{randomize()}\spxextra{spinbox.core.HilbertState method}}

\begin{fulllineitems}
\phantomsection\label{\detokenize{spinbox:spinbox.core.HilbertState.randomize}}
\pysigstartsignatures
\pysiglinewithargsret{\sphinxbfcode{\sphinxupquote{randomize}}}{\sphinxparam{\DUrole{n}{seed}\DUrole{p}{:}\DUrole{w}{ }\DUrole{n}{int}\DUrole{w}{ }\DUrole{o}{=}\DUrole{w}{ }\DUrole{default_value}{None}}}{{ $\rightarrow$ {\hyperref[\detokenize{spinbox:spinbox.core.HilbertState}]{\sphinxcrossref{HilbertState}}}}}
\pysigstopsignatures
\sphinxAtStartPar
Randomize coefficients.
\begin{quote}\begin{description}
\sphinxlineitem{Parameters}
\sphinxAtStartPar
\sphinxstyleliteralstrong{\sphinxupquote{seed}} (\sphinxstyleliteralemphasis{\sphinxupquote{int}}\sphinxstyleliteralemphasis{\sphinxupquote{, }}\sphinxstyleliteralemphasis{\sphinxupquote{optional}}) \textendash{} RNG seed, defaults to None

\sphinxlineitem{Returns}
\sphinxAtStartPar
A copy of the \sphinxcode{\sphinxupquote{HilbertState}} with random complex coefficients, normalized.

\sphinxlineitem{Return type}
\sphinxAtStartPar
{\hyperref[\detokenize{spinbox:spinbox.core.HilbertState}]{\sphinxcrossref{HilbertState}}}

\end{description}\end{quote}

\end{fulllineitems}

\index{scale() (spinbox.core.HilbertState method)@\spxentry{scale()}\spxextra{spinbox.core.HilbertState method}}

\begin{fulllineitems}
\phantomsection\label{\detokenize{spinbox:spinbox.core.HilbertState.scale}}
\pysigstartsignatures
\pysiglinewithargsret{\sphinxbfcode{\sphinxupquote{scale}}}{\sphinxparam{\DUrole{n}{other}\DUrole{p}{:}\DUrole{w}{ }\DUrole{n}{complex}}}{{ $\rightarrow$ {\hyperref[\detokenize{spinbox:spinbox.core.HilbertState}]{\sphinxcrossref{HilbertState}}}}}
\pysigstopsignatures
\sphinxAtStartPar
Scalar multiple of a \sphinxcode{\sphinxupquote{HilbertState}}.
\begin{quote}\begin{description}
\sphinxlineitem{Parameters}
\sphinxAtStartPar
\sphinxstyleliteralstrong{\sphinxupquote{other}} (\sphinxstyleliteralemphasis{\sphinxupquote{complex}}) \textendash{} Scalar number to multiply by.

\sphinxlineitem{Returns}
\sphinxAtStartPar
other * self

\sphinxlineitem{Return type}
\sphinxAtStartPar
{\hyperref[\detokenize{spinbox:spinbox.core.HilbertState}]{\sphinxcrossref{HilbertState}}}

\end{description}\end{quote}

\end{fulllineitems}

\index{zero() (spinbox.core.HilbertState method)@\spxentry{zero()}\spxextra{spinbox.core.HilbertState method}}

\begin{fulllineitems}
\phantomsection\label{\detokenize{spinbox:spinbox.core.HilbertState.zero}}
\pysigstartsignatures
\pysiglinewithargsret{\sphinxbfcode{\sphinxupquote{zero}}}{}{{ $\rightarrow$ {\hyperref[\detokenize{spinbox:spinbox.core.HilbertState}]{\sphinxcrossref{HilbertState}}}}}
\pysigstopsignatures
\sphinxAtStartPar
Set all coefficients to zero.
\begin{quote}\begin{description}
\sphinxlineitem{Returns}
\sphinxAtStartPar
A copy of \sphinxcode{\sphinxupquote{HilbertState}} with all coefficients set to zero.

\sphinxlineitem{Return type}
\sphinxAtStartPar
{\hyperref[\detokenize{spinbox:spinbox.core.HilbertState}]{\sphinxcrossref{HilbertState}}}

\end{description}\end{quote}

\end{fulllineitems}


\end{fulllineitems}

\index{Integrator (class in spinbox.core)@\spxentry{Integrator}\spxextra{class in spinbox.core}}

\begin{fulllineitems}
\phantomsection\label{\detokenize{spinbox:spinbox.core.Integrator}}
\pysigstartsignatures
\pysiglinewithargsret{\sphinxbfcode{\sphinxupquote{class\DUrole{w}{ }}}\sphinxcode{\sphinxupquote{spinbox.core.}}\sphinxbfcode{\sphinxupquote{Integrator}}}{\sphinxparam{\DUrole{n}{potential}\DUrole{p}{:}\DUrole{w}{ }\DUrole{n}{{\hyperref[\detokenize{spinbox:spinbox.core.NuclearPotential}]{\sphinxcrossref{NuclearPotential}}}}}\sphinxparamcomma \sphinxparam{\DUrole{n}{propagator}}\sphinxparamcomma \sphinxparam{\DUrole{n}{isospin}\DUrole{o}{=}\DUrole{default_value}{True}}}{}
\pysigstopsignatures
\sphinxAtStartPar
Bases: \sphinxcode{\sphinxupquote{object}}
\index{bracket() (spinbox.core.Integrator method)@\spxentry{bracket()}\spxextra{spinbox.core.Integrator method}}

\begin{fulllineitems}
\phantomsection\label{\detokenize{spinbox:spinbox.core.Integrator.bracket}}
\pysigstartsignatures
\pysiglinewithargsret{\sphinxbfcode{\sphinxupquote{bracket}}}{\sphinxparam{\DUrole{n}{bra}}\sphinxparamcomma \sphinxparam{\DUrole{n}{ket}}\sphinxparamcomma \sphinxparam{\DUrole{n}{aux\_fields}}}{}
\pysigstopsignatures
\end{fulllineitems}

\index{exact() (spinbox.core.Integrator method)@\spxentry{exact()}\spxextra{spinbox.core.Integrator method}}

\begin{fulllineitems}
\phantomsection\label{\detokenize{spinbox:spinbox.core.Integrator.exact}}
\pysigstartsignatures
\pysiglinewithargsret{\sphinxbfcode{\sphinxupquote{exact}}}{\sphinxparam{\DUrole{n}{bra}}\sphinxparamcomma \sphinxparam{\DUrole{n}{ket}}}{}
\pysigstopsignatures
\end{fulllineitems}

\index{run() (spinbox.core.Integrator method)@\spxentry{run()}\spxextra{spinbox.core.Integrator method}}

\begin{fulllineitems}
\phantomsection\label{\detokenize{spinbox:spinbox.core.Integrator.run}}
\pysigstartsignatures
\pysiglinewithargsret{\sphinxbfcode{\sphinxupquote{run}}}{\sphinxparam{\DUrole{n}{bra}}\sphinxparamcomma \sphinxparam{\DUrole{n}{ket}}}{}
\pysigstopsignatures
\end{fulllineitems}

\index{setup() (spinbox.core.Integrator method)@\spxentry{setup()}\spxextra{spinbox.core.Integrator method}}

\begin{fulllineitems}
\phantomsection\label{\detokenize{spinbox:spinbox.core.Integrator.setup}}
\pysigstartsignatures
\pysiglinewithargsret{\sphinxbfcode{\sphinxupquote{setup}}}{\sphinxparam{\DUrole{n}{n\_samples}}\sphinxparamcomma \sphinxparam{\DUrole{n}{seed}\DUrole{o}{=}\DUrole{default_value}{0}}\sphinxparamcomma \sphinxparam{\DUrole{n}{mix}\DUrole{o}{=}\DUrole{default_value}{True}}\sphinxparamcomma \sphinxparam{\DUrole{n}{flip\_aux}\DUrole{o}{=}\DUrole{default_value}{False}}\sphinxparamcomma \sphinxparam{\DUrole{n}{sigma}\DUrole{o}{=}\DUrole{default_value}{False}}\sphinxparamcomma \sphinxparam{\DUrole{n}{sigmatau}\DUrole{o}{=}\DUrole{default_value}{False}}\sphinxparamcomma \sphinxparam{\DUrole{n}{tau}\DUrole{o}{=}\DUrole{default_value}{False}}\sphinxparamcomma \sphinxparam{\DUrole{n}{coulomb}\DUrole{o}{=}\DUrole{default_value}{False}}\sphinxparamcomma \sphinxparam{\DUrole{n}{spinorbit}\DUrole{o}{=}\DUrole{default_value}{False}}\sphinxparamcomma \sphinxparam{\DUrole{n}{sigma\_3b}\DUrole{o}{=}\DUrole{default_value}{False}}\sphinxparamcomma \sphinxparam{\DUrole{n}{parallel}\DUrole{o}{=}\DUrole{default_value}{True}}\sphinxparamcomma \sphinxparam{\DUrole{n}{n\_processes}\DUrole{o}{=}\DUrole{default_value}{None}}}{}
\pysigstopsignatures
\end{fulllineitems}


\end{fulllineitems}

\index{NuclearPotential (class in spinbox.core)@\spxentry{NuclearPotential}\spxextra{class in spinbox.core}}

\begin{fulllineitems}
\phantomsection\label{\detokenize{spinbox:spinbox.core.NuclearPotential}}
\pysigstartsignatures
\pysiglinewithargsret{\sphinxbfcode{\sphinxupquote{class\DUrole{w}{ }}}\sphinxcode{\sphinxupquote{spinbox.core.}}\sphinxbfcode{\sphinxupquote{NuclearPotential}}}{\sphinxparam{\DUrole{n}{n\_particles}}}{}
\pysigstopsignatures
\sphinxAtStartPar
Bases: \sphinxcode{\sphinxupquote{object}}

\sphinxAtStartPar
container class for Argonne\sphinxhyphen{}style NN potential + NNN
\index{read\_coulomb() (spinbox.core.NuclearPotential method)@\spxentry{read\_coulomb()}\spxextra{spinbox.core.NuclearPotential method}}

\begin{fulllineitems}
\phantomsection\label{\detokenize{spinbox:spinbox.core.NuclearPotential.read_coulomb}}
\pysigstartsignatures
\pysiglinewithargsret{\sphinxbfcode{\sphinxupquote{read\_coulomb}}}{\sphinxparam{\DUrole{n}{filename}}}{}
\pysigstopsignatures
\end{fulllineitems}

\index{read\_sigma() (spinbox.core.NuclearPotential method)@\spxentry{read\_sigma()}\spxextra{spinbox.core.NuclearPotential method}}

\begin{fulllineitems}
\phantomsection\label{\detokenize{spinbox:spinbox.core.NuclearPotential.read_sigma}}
\pysigstartsignatures
\pysiglinewithargsret{\sphinxbfcode{\sphinxupquote{read\_sigma}}}{\sphinxparam{\DUrole{n}{filename}}}{}
\pysigstopsignatures
\end{fulllineitems}

\index{read\_sigma\_3b() (spinbox.core.NuclearPotential method)@\spxentry{read\_sigma\_3b()}\spxextra{spinbox.core.NuclearPotential method}}

\begin{fulllineitems}
\phantomsection\label{\detokenize{spinbox:spinbox.core.NuclearPotential.read_sigma_3b}}
\pysigstartsignatures
\pysiglinewithargsret{\sphinxbfcode{\sphinxupquote{read\_sigma\_3b}}}{\sphinxparam{\DUrole{n}{filename}}}{}
\pysigstopsignatures
\end{fulllineitems}

\index{read\_sigmatau() (spinbox.core.NuclearPotential method)@\spxentry{read\_sigmatau()}\spxextra{spinbox.core.NuclearPotential method}}

\begin{fulllineitems}
\phantomsection\label{\detokenize{spinbox:spinbox.core.NuclearPotential.read_sigmatau}}
\pysigstartsignatures
\pysiglinewithargsret{\sphinxbfcode{\sphinxupquote{read\_sigmatau}}}{\sphinxparam{\DUrole{n}{filename}}}{}
\pysigstopsignatures
\end{fulllineitems}

\index{read\_spinorbit() (spinbox.core.NuclearPotential method)@\spxentry{read\_spinorbit()}\spxextra{spinbox.core.NuclearPotential method}}

\begin{fulllineitems}
\phantomsection\label{\detokenize{spinbox:spinbox.core.NuclearPotential.read_spinorbit}}
\pysigstartsignatures
\pysiglinewithargsret{\sphinxbfcode{\sphinxupquote{read\_spinorbit}}}{\sphinxparam{\DUrole{n}{filename}}}{}
\pysigstopsignatures
\end{fulllineitems}

\index{read\_tau() (spinbox.core.NuclearPotential method)@\spxentry{read\_tau()}\spxextra{spinbox.core.NuclearPotential method}}

\begin{fulllineitems}
\phantomsection\label{\detokenize{spinbox:spinbox.core.NuclearPotential.read_tau}}
\pysigstartsignatures
\pysiglinewithargsret{\sphinxbfcode{\sphinxupquote{read\_tau}}}{\sphinxparam{\DUrole{n}{filename}}}{}
\pysigstopsignatures
\end{fulllineitems}


\end{fulllineitems}

\index{ProductOperator (class in spinbox.core)@\spxentry{ProductOperator}\spxextra{class in spinbox.core}}

\begin{fulllineitems}
\phantomsection\label{\detokenize{spinbox:spinbox.core.ProductOperator}}
\pysigstartsignatures
\pysiglinewithargsret{\sphinxbfcode{\sphinxupquote{class\DUrole{w}{ }}}\sphinxcode{\sphinxupquote{spinbox.core.}}\sphinxbfcode{\sphinxupquote{ProductOperator}}}{\sphinxparam{\DUrole{n}{n\_particles}\DUrole{p}{:}\DUrole{w}{ }\DUrole{n}{int}}\sphinxparamcomma \sphinxparam{\DUrole{n}{isospin}\DUrole{o}{=}\DUrole{default_value}{True}}}{}
\pysigstopsignatures
\sphinxAtStartPar
Bases: \sphinxcode{\sphinxupquote{object}}

\sphinxAtStartPar
An operator that is a tensor product of one\sphinxhyphen{}body operators.

\sphinxAtStartPar
As with \sphinxcode{\sphinxupquote{ProductState}} instances, \sphinxcode{\sphinxupquote{ProductOperator}} instances cannot be added or subtracted.
\index{apply\_onebody\_operator() (spinbox.core.ProductOperator method)@\spxentry{apply\_onebody\_operator()}\spxextra{spinbox.core.ProductOperator method}}

\begin{fulllineitems}
\phantomsection\label{\detokenize{spinbox:spinbox.core.ProductOperator.apply_onebody_operator}}
\pysigstartsignatures
\pysiglinewithargsret{\sphinxbfcode{\sphinxupquote{apply\_onebody\_operator}}}{\sphinxparam{\DUrole{n}{particle\_index}\DUrole{p}{:}\DUrole{w}{ }\DUrole{n}{int}}\sphinxparamcomma \sphinxparam{\DUrole{n}{spin\_matrix}\DUrole{p}{:}\DUrole{w}{ }\DUrole{n}{ndarray}}\sphinxparamcomma \sphinxparam{\DUrole{n}{isospin\_matrix}\DUrole{p}{:}\DUrole{w}{ }\DUrole{n}{ndarray}\DUrole{w}{ }\DUrole{o}{=}\DUrole{w}{ }\DUrole{default_value}{None}}}{{ $\rightarrow$ {\hyperref[\detokenize{spinbox:spinbox.core.ProductOperator}]{\sphinxcrossref{ProductOperator}}}}}
\pysigstopsignatures
\sphinxAtStartPar
Applies a one\sphinxhyphen{}body / single\sphinxhyphen{}particle operator to the \sphinxcode{\sphinxupquote{ProductOperator}}.
This accounts for the spin\sphinxhyphen{}isospin kronecker product, if isospin is used.
\begin{equation*}
\begin{split}O' = \sigma_{\alpha i} \tau_{\beta i} O\end{split}
\end{equation*}\begin{quote}\begin{description}
\sphinxlineitem{Parameters}\begin{itemize}
\item {} 
\sphinxAtStartPar
\sphinxstyleliteralstrong{\sphinxupquote{particle\_index}} (\sphinxstyleliteralemphasis{\sphinxupquote{int}}) \textendash{} Index of particle to apply onebody operator to, starting from 0.

\item {} 
\sphinxAtStartPar
\sphinxstyleliteralstrong{\sphinxupquote{spin\_matrix}} (\sphinxstyleliteralemphasis{\sphinxupquote{np.ndarray}}) \textendash{} The spin part of the operator, a 2x2 matrix

\item {} 
\sphinxAtStartPar
\sphinxstyleliteralstrong{\sphinxupquote{isospin\_matrix}} (\sphinxstyleliteralemphasis{\sphinxupquote{numpy.ndarray}}\sphinxstyleliteralemphasis{\sphinxupquote{, }}\sphinxstyleliteralemphasis{\sphinxupquote{optional}}) \textendash{} The isospin part of the operator, a 2x2 matrix, defaults to None

\end{itemize}

\sphinxlineitem{Returns}
\sphinxAtStartPar
A copy of the \sphinxcode{\sphinxupquote{ProductOperator}} with the one\sphinxhyphen{}body operator applied.

\sphinxlineitem{Return type}
\sphinxAtStartPar
{\hyperref[\detokenize{spinbox:spinbox.core.ProductOperator}]{\sphinxcrossref{ProductOperator}}}

\end{description}\end{quote}

\end{fulllineitems}

\index{apply\_sigma() (spinbox.core.ProductOperator method)@\spxentry{apply\_sigma()}\spxextra{spinbox.core.ProductOperator method}}

\begin{fulllineitems}
\phantomsection\label{\detokenize{spinbox:spinbox.core.ProductOperator.apply_sigma}}
\pysigstartsignatures
\pysiglinewithargsret{\sphinxbfcode{\sphinxupquote{apply\_sigma}}}{\sphinxparam{\DUrole{n}{particle\_index}\DUrole{p}{:}\DUrole{w}{ }\DUrole{n}{int}}\sphinxparamcomma \sphinxparam{\DUrole{n}{dimension}\DUrole{p}{:}\DUrole{w}{ }\DUrole{n}{int}}}{{ $\rightarrow$ {\hyperref[\detokenize{spinbox:spinbox.core.ProductOperator}]{\sphinxcrossref{ProductOperator}}}}}
\pysigstopsignatures
\sphinxAtStartPar
Applies a one\sphinxhyphen{}body sigma spin operator.
\begin{quote}\begin{description}
\sphinxlineitem{Parameters}\begin{itemize}
\item {} 
\sphinxAtStartPar
\sphinxstyleliteralstrong{\sphinxupquote{particle\_index}} (\sphinxstyleliteralemphasis{\sphinxupquote{int}}) \textendash{} Index of particle, staring from 0.

\item {} 
\sphinxAtStartPar
\sphinxstyleliteralstrong{\sphinxupquote{dimension}} (\sphinxstyleliteralemphasis{\sphinxupquote{int}}) \textendash{} Dimension of sigma operator: 0, 1, 2 = x, y, z

\end{itemize}

\sphinxlineitem{Returns}
\sphinxAtStartPar
The resulting \sphinxcode{\sphinxupquote{ProductOperator}}.

\sphinxlineitem{Return type}
\sphinxAtStartPar
{\hyperref[\detokenize{spinbox:spinbox.core.ProductOperator}]{\sphinxcrossref{ProductOperator}}}

\end{description}\end{quote}

\end{fulllineitems}

\index{apply\_tau() (spinbox.core.ProductOperator method)@\spxentry{apply\_tau()}\spxextra{spinbox.core.ProductOperator method}}

\begin{fulllineitems}
\phantomsection\label{\detokenize{spinbox:spinbox.core.ProductOperator.apply_tau}}
\pysigstartsignatures
\pysiglinewithargsret{\sphinxbfcode{\sphinxupquote{apply\_tau}}}{\sphinxparam{\DUrole{n}{particle\_index}\DUrole{p}{:}\DUrole{w}{ }\DUrole{n}{int}}\sphinxparamcomma \sphinxparam{\DUrole{n}{dimension}\DUrole{p}{:}\DUrole{w}{ }\DUrole{n}{int}}}{{ $\rightarrow$ {\hyperref[\detokenize{spinbox:spinbox.core.ProductOperator}]{\sphinxcrossref{ProductOperator}}}}}
\pysigstopsignatures
\sphinxAtStartPar
Applies a one\sphinxhyphen{}body tau isospin operator.
\begin{quote}\begin{description}
\sphinxlineitem{Parameters}\begin{itemize}
\item {} 
\sphinxAtStartPar
\sphinxstyleliteralstrong{\sphinxupquote{particle\_index}} (\sphinxstyleliteralemphasis{\sphinxupquote{int}}) \textendash{} Index of particle, staring from 0.

\item {} 
\sphinxAtStartPar
\sphinxstyleliteralstrong{\sphinxupquote{dimension}} (\sphinxstyleliteralemphasis{\sphinxupquote{int}}) \textendash{} Dimension of tau operator: 0, 1, 2 = x, y, z

\end{itemize}

\sphinxlineitem{Returns}
\sphinxAtStartPar
The resulting \sphinxcode{\sphinxupquote{ProductOperator}}.

\sphinxlineitem{Return type}
\sphinxAtStartPar
{\hyperref[\detokenize{spinbox:spinbox.core.ProductOperator}]{\sphinxcrossref{ProductOperator}}}

\end{description}\end{quote}

\end{fulllineitems}

\index{copy() (spinbox.core.ProductOperator method)@\spxentry{copy()}\spxextra{spinbox.core.ProductOperator method}}

\begin{fulllineitems}
\phantomsection\label{\detokenize{spinbox:spinbox.core.ProductOperator.copy}}
\pysigstartsignatures
\pysiglinewithargsret{\sphinxbfcode{\sphinxupquote{copy}}}{}{{ $\rightarrow$ {\hyperref[\detokenize{spinbox:spinbox.core.ProductOperator}]{\sphinxcrossref{ProductOperator}}}}}
\pysigstopsignatures
\sphinxAtStartPar
Copies the \sphinxcode{\sphinxupquote{ProductOperator}}.
\begin{quote}\begin{description}
\sphinxlineitem{Returns}
\sphinxAtStartPar
a new instance of \sphinxcode{\sphinxupquote{ProductOperator}} with all the same properties as self.

\sphinxlineitem{Return type}
\sphinxAtStartPar
{\hyperref[\detokenize{spinbox:spinbox.core.ProductOperator}]{\sphinxcrossref{ProductOperator}}}

\end{description}\end{quote}

\end{fulllineitems}

\index{dagger() (spinbox.core.ProductOperator method)@\spxentry{dagger()}\spxextra{spinbox.core.ProductOperator method}}

\begin{fulllineitems}
\phantomsection\label{\detokenize{spinbox:spinbox.core.ProductOperator.dagger}}
\pysigstartsignatures
\pysiglinewithargsret{\sphinxbfcode{\sphinxupquote{dagger}}}{}{}
\pysigstopsignatures
\sphinxAtStartPar
Hermitian conjugate.
\begin{quote}\begin{description}
\sphinxlineitem{Returns}
\sphinxAtStartPar
The dual \sphinxcode{\sphinxupquote{ProductOperator}}

\sphinxlineitem{Return type}
\sphinxAtStartPar
{\hyperref[\detokenize{spinbox:spinbox.core.ProductOperator}]{\sphinxcrossref{ProductOperator}}}

\end{description}\end{quote}

\end{fulllineitems}

\index{multiply\_operator() (spinbox.core.ProductOperator method)@\spxentry{multiply\_operator()}\spxextra{spinbox.core.ProductOperator method}}

\begin{fulllineitems}
\phantomsection\label{\detokenize{spinbox:spinbox.core.ProductOperator.multiply_operator}}
\pysigstartsignatures
\pysiglinewithargsret{\sphinxbfcode{\sphinxupquote{multiply\_operator}}}{\sphinxparam{\DUrole{n}{other}\DUrole{p}{:}\DUrole{w}{ }\DUrole{n}{{\hyperref[\detokenize{spinbox:spinbox.core.ProductOperator}]{\sphinxcrossref{ProductOperator}}}}}}{{ $\rightarrow$ {\hyperref[\detokenize{spinbox:spinbox.core.ProductOperator}]{\sphinxcrossref{ProductOperator}}}}}
\pysigstopsignatures
\sphinxAtStartPar
Multiply two \sphinxcode{\sphinxupquote{ProductOperator}} instances together to get a new one.
\begin{quote}\begin{description}
\sphinxlineitem{Parameters}
\sphinxAtStartPar
\sphinxstyleliteralstrong{\sphinxupquote{other}} ({\hyperref[\detokenize{spinbox:spinbox.core.ProductOperator}]{\sphinxcrossref{\sphinxstyleliteralemphasis{\sphinxupquote{ProductOperator}}}}}) \textendash{} The other \sphinxcode{\sphinxupquote{ProductOperator}}

\sphinxlineitem{Returns}
\sphinxAtStartPar
The product of the two.

\sphinxlineitem{Return type}
\sphinxAtStartPar
{\hyperref[\detokenize{spinbox:spinbox.core.ProductOperator}]{\sphinxcrossref{ProductOperator}}}

\end{description}\end{quote}

\end{fulllineitems}

\index{multiply\_state() (spinbox.core.ProductOperator method)@\spxentry{multiply\_state()}\spxextra{spinbox.core.ProductOperator method}}

\begin{fulllineitems}
\phantomsection\label{\detokenize{spinbox:spinbox.core.ProductOperator.multiply_state}}
\pysigstartsignatures
\pysiglinewithargsret{\sphinxbfcode{\sphinxupquote{multiply\_state}}}{\sphinxparam{\DUrole{n}{other}\DUrole{p}{:}\DUrole{w}{ }\DUrole{n}{{\hyperref[\detokenize{spinbox:spinbox.core.ProductState}]{\sphinxcrossref{ProductState}}}}}}{{ $\rightarrow$ {\hyperref[\detokenize{spinbox:spinbox.core.ProductState}]{\sphinxcrossref{ProductState}}}}}
\pysigstopsignatures
\sphinxAtStartPar
Apply the operator to a \sphinxcode{\sphinxupquote{ProductState}} ket.
\begin{quote}\begin{description}
\sphinxlineitem{Parameters}
\sphinxAtStartPar
\sphinxstyleliteralstrong{\sphinxupquote{other}} ({\hyperref[\detokenize{spinbox:spinbox.core.ProductState}]{\sphinxcrossref{\sphinxstyleliteralemphasis{\sphinxupquote{ProductState}}}}}) \textendash{} The state, ketwise.

\sphinxlineitem{Returns}
\sphinxAtStartPar
The new state, ketwise.

\sphinxlineitem{Return type}
\sphinxAtStartPar
{\hyperref[\detokenize{spinbox:spinbox.core.ProductState}]{\sphinxcrossref{ProductState}}}

\end{description}\end{quote}

\end{fulllineitems}

\index{scale\_all() (spinbox.core.ProductOperator method)@\spxentry{scale\_all()}\spxextra{spinbox.core.ProductOperator method}}

\begin{fulllineitems}
\phantomsection\label{\detokenize{spinbox:spinbox.core.ProductOperator.scale_all}}
\pysigstartsignatures
\pysiglinewithargsret{\sphinxbfcode{\sphinxupquote{scale\_all}}}{\sphinxparam{\DUrole{n}{b}}}{{ $\rightarrow$ {\hyperref[\detokenize{spinbox:spinbox.core.ProductOperator}]{\sphinxcrossref{ProductOperator}}}}}
\pysigstopsignatures
\sphinxAtStartPar
Scales an A\sphinxhyphen{}body operator by \sphinxcode{\sphinxupquote{b}} by multiplying each one\sphinxhyphen{}body matrix by the Ath root of \sphinxcode{\sphinxupquote{b}}.
\begin{quote}\begin{description}
\sphinxlineitem{Parameters}
\sphinxAtStartPar
\sphinxstyleliteralstrong{\sphinxupquote{b}} (\sphinxstyleliteralemphasis{\sphinxupquote{complex}}) \textendash{} scalar

\sphinxlineitem{Returns}
\sphinxAtStartPar
The scaled state

\sphinxlineitem{Return type}
\sphinxAtStartPar
{\hyperref[\detokenize{spinbox:spinbox.core.ProductOperator}]{\sphinxcrossref{ProductOperator}}}

\end{description}\end{quote}

\end{fulllineitems}

\index{scale\_one() (spinbox.core.ProductOperator method)@\spxentry{scale\_one()}\spxextra{spinbox.core.ProductOperator method}}

\begin{fulllineitems}
\phantomsection\label{\detokenize{spinbox:spinbox.core.ProductOperator.scale_one}}
\pysigstartsignatures
\pysiglinewithargsret{\sphinxbfcode{\sphinxupquote{scale\_one}}}{\sphinxparam{\DUrole{n}{particle\_index}\DUrole{p}{:}\DUrole{w}{ }\DUrole{n}{int}}\sphinxparamcomma \sphinxparam{\DUrole{n}{b}\DUrole{p}{:}\DUrole{w}{ }\DUrole{n}{complex}}}{{ $\rightarrow$ {\hyperref[\detokenize{spinbox:spinbox.core.ProductOperator}]{\sphinxcrossref{ProductOperator}}}}}
\pysigstopsignatures
\sphinxAtStartPar
Multiplies a single particle operator matrix by a number.
\begin{quote}\begin{description}
\sphinxlineitem{Parameters}\begin{itemize}
\item {} 
\sphinxAtStartPar
\sphinxstyleliteralstrong{\sphinxupquote{particle\_index}} (\sphinxstyleliteralemphasis{\sphinxupquote{int}}) \textendash{} Index of particle, starting from 0.

\item {} 
\sphinxAtStartPar
\sphinxstyleliteralstrong{\sphinxupquote{b}} (\sphinxstyleliteralemphasis{\sphinxupquote{complex}}) \textendash{} Scalar

\end{itemize}

\sphinxlineitem{Returns}
\sphinxAtStartPar
A copy of the \sphinxcode{\sphinxupquote{ProductOperator}} with the one\sphinxhyphen{}body matrix scaled.

\sphinxlineitem{Return type}
\sphinxAtStartPar
{\hyperref[\detokenize{spinbox:spinbox.core.ProductOperator}]{\sphinxcrossref{ProductOperator}}}

\end{description}\end{quote}

\end{fulllineitems}

\index{to\_list() (spinbox.core.ProductOperator method)@\spxentry{to\_list()}\spxextra{spinbox.core.ProductOperator method}}

\begin{fulllineitems}
\phantomsection\label{\detokenize{spinbox:spinbox.core.ProductOperator.to_list}}
\pysigstartsignatures
\pysiglinewithargsret{\sphinxbfcode{\sphinxupquote{to\_list}}}{}{{ $\rightarrow$ list\DUrole{p}{{[}}ndarray\DUrole{p}{{]}}}}
\pysigstopsignatures\begin{quote}\begin{description}
\sphinxlineitem{Returns}
\sphinxAtStartPar
A list of one\sphinxhyphen{}body operator matrices

\sphinxlineitem{Return type}
\sphinxAtStartPar
list{[}numpy.ndarray{]}

\end{description}\end{quote}

\end{fulllineitems}

\index{to\_manybody\_basis() (spinbox.core.ProductOperator method)@\spxentry{to\_manybody\_basis()}\spxextra{spinbox.core.ProductOperator method}}

\begin{fulllineitems}
\phantomsection\label{\detokenize{spinbox:spinbox.core.ProductOperator.to_manybody_basis}}
\pysigstartsignatures
\pysiglinewithargsret{\sphinxbfcode{\sphinxupquote{to\_manybody\_basis}}}{}{}
\pysigstopsignatures
\sphinxAtStartPar
Projects to the many\sphinxhyphen{}body basis.
\begin{quote}\begin{description}
\sphinxlineitem{Returns}
\sphinxAtStartPar
The Kronecker product of the \sphinxcode{\sphinxupquote{ProductOperator}}.

\sphinxlineitem{Return type}
\sphinxAtStartPar
{\hyperref[\detokenize{spinbox:spinbox.core.HilbertOperator}]{\sphinxcrossref{HilbertOperator}}}

\end{description}\end{quote}

\end{fulllineitems}

\index{zero() (spinbox.core.ProductOperator method)@\spxentry{zero()}\spxextra{spinbox.core.ProductOperator method}}

\begin{fulllineitems}
\phantomsection\label{\detokenize{spinbox:spinbox.core.ProductOperator.zero}}
\pysigstartsignatures
\pysiglinewithargsret{\sphinxbfcode{\sphinxupquote{zero}}}{}{{ $\rightarrow$ {\hyperref[\detokenize{spinbox:spinbox.core.ProductOperator}]{\sphinxcrossref{ProductOperator}}}}}
\pysigstopsignatures
\sphinxAtStartPar
Set all coefficients to zero.
\begin{quote}\begin{description}
\sphinxlineitem{Returns}
\sphinxAtStartPar
A copy of \sphinxcode{\sphinxupquote{ProductOperator}} with all coefficients set to zero.

\sphinxlineitem{Return type}
\sphinxAtStartPar
{\hyperref[\detokenize{spinbox:spinbox.core.ProductOperator}]{\sphinxcrossref{ProductOperator}}}

\end{description}\end{quote}

\end{fulllineitems}


\end{fulllineitems}

\index{ProductPropagatorHS (class in spinbox.core)@\spxentry{ProductPropagatorHS}\spxextra{class in spinbox.core}}

\begin{fulllineitems}
\phantomsection\label{\detokenize{spinbox:spinbox.core.ProductPropagatorHS}}
\pysigstartsignatures
\pysiglinewithargsret{\sphinxbfcode{\sphinxupquote{class\DUrole{w}{ }}}\sphinxcode{\sphinxupquote{spinbox.core.}}\sphinxbfcode{\sphinxupquote{ProductPropagatorHS}}}{\sphinxparam{\DUrole{n}{n\_particles}\DUrole{p}{:}\DUrole{w}{ }\DUrole{n}{int}}\sphinxparamcomma \sphinxparam{\DUrole{n}{dt}\DUrole{p}{:}\DUrole{w}{ }\DUrole{n}{float}}\sphinxparamcomma \sphinxparam{\DUrole{n}{isospin}\DUrole{o}{=}\DUrole{default_value}{True}}\sphinxparamcomma \sphinxparam{\DUrole{n}{include\_prefactors}\DUrole{o}{=}\DUrole{default_value}{True}}}{}
\pysigstopsignatures
\sphinxAtStartPar
Bases: {\hyperref[\detokenize{spinbox:spinbox.core.Propagator}]{\sphinxcrossref{\sphinxcode{\sphinxupquote{Propagator}}}}}

\sphinxAtStartPar
the propagator exp( \sphinxhyphen{} i z op\_i op\_j )
\index{factors\_coulomb() (spinbox.core.ProductPropagatorHS method)@\spxentry{factors\_coulomb()}\spxextra{spinbox.core.ProductPropagatorHS method}}

\begin{fulllineitems}
\phantomsection\label{\detokenize{spinbox:spinbox.core.ProductPropagatorHS.factors_coulomb}}
\pysigstartsignatures
\pysiglinewithargsret{\sphinxbfcode{\sphinxupquote{factors\_coulomb}}}{\sphinxparam{\DUrole{n}{coupling}\DUrole{p}{:}\DUrole{w}{ }\DUrole{n}{{\hyperref[\detokenize{spinbox:spinbox.core.CoulombCoupling}]{\sphinxcrossref{CoulombCoupling}}}}}\sphinxparamcomma \sphinxparam{\DUrole{n}{aux}\DUrole{p}{:}\DUrole{w}{ }\DUrole{n}{list}}}{}
\pysigstopsignatures
\end{fulllineitems}

\index{factors\_sigma() (spinbox.core.ProductPropagatorHS method)@\spxentry{factors\_sigma()}\spxextra{spinbox.core.ProductPropagatorHS method}}

\begin{fulllineitems}
\phantomsection\label{\detokenize{spinbox:spinbox.core.ProductPropagatorHS.factors_sigma}}
\pysigstartsignatures
\pysiglinewithargsret{\sphinxbfcode{\sphinxupquote{factors\_sigma}}}{\sphinxparam{\DUrole{n}{coupling}\DUrole{p}{:}\DUrole{w}{ }\DUrole{n}{{\hyperref[\detokenize{spinbox:spinbox.core.SigmaCoupling}]{\sphinxcrossref{SigmaCoupling}}}}}\sphinxparamcomma \sphinxparam{\DUrole{n}{aux}\DUrole{p}{:}\DUrole{w}{ }\DUrole{n}{list}}}{}
\pysigstopsignatures
\end{fulllineitems}

\index{factors\_sigmatau() (spinbox.core.ProductPropagatorHS method)@\spxentry{factors\_sigmatau()}\spxextra{spinbox.core.ProductPropagatorHS method}}

\begin{fulllineitems}
\phantomsection\label{\detokenize{spinbox:spinbox.core.ProductPropagatorHS.factors_sigmatau}}
\pysigstartsignatures
\pysiglinewithargsret{\sphinxbfcode{\sphinxupquote{factors\_sigmatau}}}{\sphinxparam{\DUrole{n}{coupling}\DUrole{p}{:}\DUrole{w}{ }\DUrole{n}{{\hyperref[\detokenize{spinbox:spinbox.core.SigmaTauCoupling}]{\sphinxcrossref{SigmaTauCoupling}}}}}\sphinxparamcomma \sphinxparam{\DUrole{n}{aux}\DUrole{p}{:}\DUrole{w}{ }\DUrole{n}{list}}}{}
\pysigstopsignatures
\end{fulllineitems}

\index{factors\_spinorbit() (spinbox.core.ProductPropagatorHS method)@\spxentry{factors\_spinorbit()}\spxextra{spinbox.core.ProductPropagatorHS method}}

\begin{fulllineitems}
\phantomsection\label{\detokenize{spinbox:spinbox.core.ProductPropagatorHS.factors_spinorbit}}
\pysigstartsignatures
\pysiglinewithargsret{\sphinxbfcode{\sphinxupquote{factors\_spinorbit}}}{\sphinxparam{\DUrole{n}{coupling}\DUrole{p}{:}\DUrole{w}{ }\DUrole{n}{{\hyperref[\detokenize{spinbox:spinbox.core.SpinOrbitCoupling}]{\sphinxcrossref{SpinOrbitCoupling}}}}}\sphinxparamcomma \sphinxparam{\DUrole{n}{aux}\DUrole{p}{:}\DUrole{w}{ }\DUrole{n}{list}}}{}
\pysigstopsignatures
\end{fulllineitems}

\index{factors\_tau() (spinbox.core.ProductPropagatorHS method)@\spxentry{factors\_tau()}\spxextra{spinbox.core.ProductPropagatorHS method}}

\begin{fulllineitems}
\phantomsection\label{\detokenize{spinbox:spinbox.core.ProductPropagatorHS.factors_tau}}
\pysigstartsignatures
\pysiglinewithargsret{\sphinxbfcode{\sphinxupquote{factors\_tau}}}{\sphinxparam{\DUrole{n}{coupling}\DUrole{p}{:}\DUrole{w}{ }\DUrole{n}{{\hyperref[\detokenize{spinbox:spinbox.core.TauCoupling}]{\sphinxcrossref{TauCoupling}}}}}\sphinxparamcomma \sphinxparam{\DUrole{n}{aux}\DUrole{p}{:}\DUrole{w}{ }\DUrole{n}{list}}}{}
\pysigstopsignatures
\end{fulllineitems}

\index{onebody() (spinbox.core.ProductPropagatorHS method)@\spxentry{onebody()}\spxextra{spinbox.core.ProductPropagatorHS method}}

\begin{fulllineitems}
\phantomsection\label{\detokenize{spinbox:spinbox.core.ProductPropagatorHS.onebody}}
\pysigstartsignatures
\pysiglinewithargsret{\sphinxbfcode{\sphinxupquote{onebody}}}{\sphinxparam{\DUrole{n}{z}\DUrole{p}{:}\DUrole{w}{ }\DUrole{n}{complex}}\sphinxparamcomma \sphinxparam{\DUrole{n}{i}\DUrole{p}{:}\DUrole{w}{ }\DUrole{n}{int}}\sphinxparamcomma \sphinxparam{\DUrole{n}{onebody\_matrix}\DUrole{p}{:}\DUrole{w}{ }\DUrole{n}{ndarray}}}{}
\pysigstopsignatures
\sphinxAtStartPar
exp (\sphinxhyphen{} z opi)

\end{fulllineitems}

\index{twobody\_sample() (spinbox.core.ProductPropagatorHS method)@\spxentry{twobody\_sample()}\spxextra{spinbox.core.ProductPropagatorHS method}}

\begin{fulllineitems}
\phantomsection\label{\detokenize{spinbox:spinbox.core.ProductPropagatorHS.twobody_sample}}
\pysigstartsignatures
\pysiglinewithargsret{\sphinxbfcode{\sphinxupquote{twobody\_sample}}}{\sphinxparam{\DUrole{n}{z}\DUrole{p}{:}\DUrole{w}{ }\DUrole{n}{complex}}\sphinxparamcomma \sphinxparam{\DUrole{n}{x}\DUrole{p}{:}\DUrole{w}{ }\DUrole{n}{float}}\sphinxparamcomma \sphinxparam{\DUrole{n}{i}\DUrole{p}{:}\DUrole{w}{ }\DUrole{n}{int}}\sphinxparamcomma \sphinxparam{\DUrole{n}{j}\DUrole{p}{:}\DUrole{w}{ }\DUrole{n}{int}}\sphinxparamcomma \sphinxparam{\DUrole{n}{onebody\_matrix\_i}\DUrole{p}{:}\DUrole{w}{ }\DUrole{n}{ndarray}}\sphinxparamcomma \sphinxparam{\DUrole{n}{onebody\_matrix\_j}\DUrole{p}{:}\DUrole{w}{ }\DUrole{n}{ndarray}}}{}
\pysigstopsignatures
\sphinxAtStartPar
exp ( x * sqrt( \sphinxhyphen{}z ) opi)  exp ( x * sqrt( \sphinxhyphen{}z ) opj)

\end{fulllineitems}


\end{fulllineitems}

\index{ProductPropagatorRBM (class in spinbox.core)@\spxentry{ProductPropagatorRBM}\spxextra{class in spinbox.core}}

\begin{fulllineitems}
\phantomsection\label{\detokenize{spinbox:spinbox.core.ProductPropagatorRBM}}
\pysigstartsignatures
\pysiglinewithargsret{\sphinxbfcode{\sphinxupquote{class\DUrole{w}{ }}}\sphinxcode{\sphinxupquote{spinbox.core.}}\sphinxbfcode{\sphinxupquote{ProductPropagatorRBM}}}{\sphinxparam{\DUrole{n}{n\_particles}}\sphinxparamcomma \sphinxparam{\DUrole{n}{dt}}\sphinxparamcomma \sphinxparam{\DUrole{n}{isospin}\DUrole{o}{=}\DUrole{default_value}{True}}\sphinxparamcomma \sphinxparam{\DUrole{n}{include\_prefactors}\DUrole{o}{=}\DUrole{default_value}{True}}}{}
\pysigstopsignatures
\sphinxAtStartPar
Bases: {\hyperref[\detokenize{spinbox:spinbox.core.Propagator}]{\sphinxcrossref{\sphinxcode{\sphinxupquote{Propagator}}}}}

\sphinxAtStartPar
exp( \sphinxhyphen{} i z op\_i op\_j )
seed determines mixing
\index{factors\_coulomb() (spinbox.core.ProductPropagatorRBM method)@\spxentry{factors\_coulomb()}\spxextra{spinbox.core.ProductPropagatorRBM method}}

\begin{fulllineitems}
\phantomsection\label{\detokenize{spinbox:spinbox.core.ProductPropagatorRBM.factors_coulomb}}
\pysigstartsignatures
\pysiglinewithargsret{\sphinxbfcode{\sphinxupquote{factors\_coulomb}}}{\sphinxparam{\DUrole{n}{coupling}\DUrole{p}{:}\DUrole{w}{ }\DUrole{n}{{\hyperref[\detokenize{spinbox:spinbox.core.CoulombCoupling}]{\sphinxcrossref{CoulombCoupling}}}}}\sphinxparamcomma \sphinxparam{\DUrole{n}{aux}\DUrole{p}{:}\DUrole{w}{ }\DUrole{n}{list}}}{}
\pysigstopsignatures
\end{fulllineitems}

\index{factors\_sigma() (spinbox.core.ProductPropagatorRBM method)@\spxentry{factors\_sigma()}\spxextra{spinbox.core.ProductPropagatorRBM method}}

\begin{fulllineitems}
\phantomsection\label{\detokenize{spinbox:spinbox.core.ProductPropagatorRBM.factors_sigma}}
\pysigstartsignatures
\pysiglinewithargsret{\sphinxbfcode{\sphinxupquote{factors\_sigma}}}{\sphinxparam{\DUrole{n}{coupling}\DUrole{p}{:}\DUrole{w}{ }\DUrole{n}{{\hyperref[\detokenize{spinbox:spinbox.core.SigmaCoupling}]{\sphinxcrossref{SigmaCoupling}}}}}\sphinxparamcomma \sphinxparam{\DUrole{n}{aux}\DUrole{p}{:}\DUrole{w}{ }\DUrole{n}{list}}}{}
\pysigstopsignatures
\end{fulllineitems}

\index{factors\_sigma\_3b() (spinbox.core.ProductPropagatorRBM method)@\spxentry{factors\_sigma\_3b()}\spxextra{spinbox.core.ProductPropagatorRBM method}}

\begin{fulllineitems}
\phantomsection\label{\detokenize{spinbox:spinbox.core.ProductPropagatorRBM.factors_sigma_3b}}
\pysigstartsignatures
\pysiglinewithargsret{\sphinxbfcode{\sphinxupquote{factors\_sigma\_3b}}}{\sphinxparam{\DUrole{n}{coupling}\DUrole{p}{:}\DUrole{w}{ }\DUrole{n}{{\hyperref[\detokenize{spinbox:spinbox.core.ThreeBodyCoupling}]{\sphinxcrossref{ThreeBodyCoupling}}}}}\sphinxparamcomma \sphinxparam{\DUrole{n}{aux}\DUrole{p}{:}\DUrole{w}{ }\DUrole{n}{list}}}{}
\pysigstopsignatures
\end{fulllineitems}

\index{factors\_sigmatau() (spinbox.core.ProductPropagatorRBM method)@\spxentry{factors\_sigmatau()}\spxextra{spinbox.core.ProductPropagatorRBM method}}

\begin{fulllineitems}
\phantomsection\label{\detokenize{spinbox:spinbox.core.ProductPropagatorRBM.factors_sigmatau}}
\pysigstartsignatures
\pysiglinewithargsret{\sphinxbfcode{\sphinxupquote{factors\_sigmatau}}}{\sphinxparam{\DUrole{n}{coupling}\DUrole{p}{:}\DUrole{w}{ }\DUrole{n}{{\hyperref[\detokenize{spinbox:spinbox.core.SigmaTauCoupling}]{\sphinxcrossref{SigmaTauCoupling}}}}}\sphinxparamcomma \sphinxparam{\DUrole{n}{aux}\DUrole{p}{:}\DUrole{w}{ }\DUrole{n}{list}}}{}
\pysigstopsignatures
\end{fulllineitems}

\index{factors\_spinorbit() (spinbox.core.ProductPropagatorRBM method)@\spxentry{factors\_spinorbit()}\spxextra{spinbox.core.ProductPropagatorRBM method}}

\begin{fulllineitems}
\phantomsection\label{\detokenize{spinbox:spinbox.core.ProductPropagatorRBM.factors_spinorbit}}
\pysigstartsignatures
\pysiglinewithargsret{\sphinxbfcode{\sphinxupquote{factors\_spinorbit}}}{\sphinxparam{\DUrole{n}{coupling}\DUrole{p}{:}\DUrole{w}{ }\DUrole{n}{{\hyperref[\detokenize{spinbox:spinbox.core.SpinOrbitCoupling}]{\sphinxcrossref{SpinOrbitCoupling}}}}}\sphinxparamcomma \sphinxparam{\DUrole{n}{aux}\DUrole{p}{:}\DUrole{w}{ }\DUrole{n}{list}}}{}
\pysigstopsignatures
\end{fulllineitems}

\index{factors\_tau() (spinbox.core.ProductPropagatorRBM method)@\spxentry{factors\_tau()}\spxextra{spinbox.core.ProductPropagatorRBM method}}

\begin{fulllineitems}
\phantomsection\label{\detokenize{spinbox:spinbox.core.ProductPropagatorRBM.factors_tau}}
\pysigstartsignatures
\pysiglinewithargsret{\sphinxbfcode{\sphinxupquote{factors\_tau}}}{\sphinxparam{\DUrole{n}{coupling}\DUrole{p}{:}\DUrole{w}{ }\DUrole{n}{{\hyperref[\detokenize{spinbox:spinbox.core.TauCoupling}]{\sphinxcrossref{TauCoupling}}}}}\sphinxparamcomma \sphinxparam{\DUrole{n}{aux}\DUrole{p}{:}\DUrole{w}{ }\DUrole{n}{list}}}{}
\pysigstopsignatures
\end{fulllineitems}

\index{onebody() (spinbox.core.ProductPropagatorRBM method)@\spxentry{onebody()}\spxextra{spinbox.core.ProductPropagatorRBM method}}

\begin{fulllineitems}
\phantomsection\label{\detokenize{spinbox:spinbox.core.ProductPropagatorRBM.onebody}}
\pysigstartsignatures
\pysiglinewithargsret{\sphinxbfcode{\sphinxupquote{onebody}}}{\sphinxparam{\DUrole{n}{z}\DUrole{p}{:}\DUrole{w}{ }\DUrole{n}{complex}}\sphinxparamcomma \sphinxparam{\DUrole{n}{i}\DUrole{p}{:}\DUrole{w}{ }\DUrole{n}{int}}\sphinxparamcomma \sphinxparam{\DUrole{n}{onebody\_matrix}\DUrole{p}{:}\DUrole{w}{ }\DUrole{n}{ndarray}}}{}
\pysigstopsignatures
\sphinxAtStartPar
exp (\sphinxhyphen{} i z opi)

\end{fulllineitems}

\index{threebody\_sample() (spinbox.core.ProductPropagatorRBM method)@\spxentry{threebody\_sample()}\spxextra{spinbox.core.ProductPropagatorRBM method}}

\begin{fulllineitems}
\phantomsection\label{\detokenize{spinbox:spinbox.core.ProductPropagatorRBM.threebody_sample}}
\pysigstartsignatures
\pysiglinewithargsret{\sphinxbfcode{\sphinxupquote{threebody\_sample}}}{\sphinxparam{\DUrole{n}{z}\DUrole{p}{:}\DUrole{w}{ }\DUrole{n}{float}}\sphinxparamcomma \sphinxparam{\DUrole{n}{h\_list}\DUrole{p}{:}\DUrole{w}{ }\DUrole{n}{list}}\sphinxparamcomma \sphinxparam{\DUrole{n}{i}\DUrole{p}{:}\DUrole{w}{ }\DUrole{n}{int}}\sphinxparamcomma \sphinxparam{\DUrole{n}{j}\DUrole{p}{:}\DUrole{w}{ }\DUrole{n}{int}}\sphinxparamcomma \sphinxparam{\DUrole{n}{k}\DUrole{p}{:}\DUrole{w}{ }\DUrole{n}{int}}\sphinxparamcomma \sphinxparam{\DUrole{n}{onebody\_matrix\_i}}\sphinxparamcomma \sphinxparam{\DUrole{n}{onebody\_matrix\_j}}\sphinxparamcomma \sphinxparam{\DUrole{n}{onebody\_matrix\_k}}}{}
\pysigstopsignatures
\sphinxAtStartPar
three body RBM sample written for one combined 3\sphinxhyphen{}body RBM kernel function

\end{fulllineitems}

\index{twobody\_sample() (spinbox.core.ProductPropagatorRBM method)@\spxentry{twobody\_sample()}\spxextra{spinbox.core.ProductPropagatorRBM method}}

\begin{fulllineitems}
\phantomsection\label{\detokenize{spinbox:spinbox.core.ProductPropagatorRBM.twobody_sample}}
\pysigstartsignatures
\pysiglinewithargsret{\sphinxbfcode{\sphinxupquote{twobody\_sample}}}{\sphinxparam{\DUrole{n}{z}\DUrole{p}{:}\DUrole{w}{ }\DUrole{n}{complex}}\sphinxparamcomma \sphinxparam{\DUrole{n}{h}\DUrole{p}{:}\DUrole{w}{ }\DUrole{n}{int}}\sphinxparamcomma \sphinxparam{\DUrole{n}{i}\DUrole{p}{:}\DUrole{w}{ }\DUrole{n}{int}}\sphinxparamcomma \sphinxparam{\DUrole{n}{j}\DUrole{p}{:}\DUrole{w}{ }\DUrole{n}{int}}\sphinxparamcomma \sphinxparam{\DUrole{n}{onebody\_matrix\_i}}\sphinxparamcomma \sphinxparam{\DUrole{n}{onebody\_matrix\_j}}}{}
\pysigstopsignatures
\end{fulllineitems}


\end{fulllineitems}

\index{ProductState (class in spinbox.core)@\spxentry{ProductState}\spxextra{class in spinbox.core}}

\begin{fulllineitems}
\phantomsection\label{\detokenize{spinbox:spinbox.core.ProductState}}
\pysigstartsignatures
\pysiglinewithargsret{\sphinxbfcode{\sphinxupquote{class\DUrole{w}{ }}}\sphinxcode{\sphinxupquote{spinbox.core.}}\sphinxbfcode{\sphinxupquote{ProductState}}}{\sphinxparam{\DUrole{n}{n\_particles}\DUrole{p}{:}\DUrole{w}{ }\DUrole{n}{int}}\sphinxparamcomma \sphinxparam{\DUrole{n}{coefficients}\DUrole{o}{=}\DUrole{default_value}{None}}\sphinxparamcomma \sphinxparam{\DUrole{n}{ketwise}\DUrole{o}{=}\DUrole{default_value}{True}}\sphinxparamcomma \sphinxparam{\DUrole{n}{isospin}\DUrole{o}{=}\DUrole{default_value}{True}}}{}
\pysigstopsignatures
\sphinxAtStartPar
Bases: \sphinxcode{\sphinxupquote{object}}

\sphinxAtStartPar
A spin state in the “Product” basis, a single tensor product of one\sphinxhyphen{}body vectors.

\sphinxAtStartPar
States must be defined with a number of particles. 
If \sphinxcode{\sphinxupquote{isospin}} is False, then the one\sphinxhyphen{}body basis is only spin up/down. If True, then it is (spin up/down x isospin up/down).
\sphinxcode{\sphinxupquote{ketwise}} detemines if it is a bra or a ket.

\sphinxAtStartPar
Tensor product states do not form a proper vector space (e.g. the sum of two is not guaranteed to be a tensor product)
so methods with \sphinxcode{\sphinxupquote{ProductState}} are restricted. Namely operations + and \sphinxhyphen{} do not exist.

\sphinxAtStartPar
The coefficients of the \sphinxcode{\sphinxupquote{ProductState}} are kept in the one\sphinxhyphen{}body form and can be projected to the Hilbert basis using the \sphinxcode{\sphinxupquote{to\_manybody\_basis}} method.
\index{attach\_coordinates() (spinbox.core.ProductState method)@\spxentry{attach\_coordinates()}\spxextra{spinbox.core.ProductState method}}

\begin{fulllineitems}
\phantomsection\label{\detokenize{spinbox:spinbox.core.ProductState.attach_coordinates}}
\pysigstartsignatures
\pysiglinewithargsret{\sphinxbfcode{\sphinxupquote{attach\_coordinates}}}{\sphinxparam{\DUrole{n}{coordinates}\DUrole{p}{:}\DUrole{w}{ }\DUrole{n}{ndarray}}}{}
\pysigstopsignatures
\sphinxAtStartPar
Adds a new \sphinxcode{\sphinxupquote{.coordinates}} attribute to the \sphinxcode{\sphinxupquote{ProductState}}
\begin{quote}\begin{description}
\sphinxlineitem{Parameters}
\sphinxAtStartPar
\sphinxstyleliteralstrong{\sphinxupquote{coordinates}} (\sphinxstyleliteralemphasis{\sphinxupquote{np.ndarray}}) \textendash{} A Numpy array with shape \sphinxcode{\sphinxupquote{(n\_particles , 3)}} (e.g. x, y, z)

\end{description}\end{quote}

\end{fulllineitems}

\index{copy() (spinbox.core.ProductState method)@\spxentry{copy()}\spxextra{spinbox.core.ProductState method}}

\begin{fulllineitems}
\phantomsection\label{\detokenize{spinbox:spinbox.core.ProductState.copy}}
\pysigstartsignatures
\pysiglinewithargsret{\sphinxbfcode{\sphinxupquote{copy}}}{}{}
\pysigstopsignatures
\sphinxAtStartPar
Copies the \sphinxcode{\sphinxupquote{ProductState}}.
\begin{quote}\begin{description}
\sphinxlineitem{Returns}
\sphinxAtStartPar
a new instance of \sphinxcode{\sphinxupquote{ProductState}} with all the same properties.

\sphinxlineitem{Return type}
\sphinxAtStartPar
{\hyperref[\detokenize{spinbox:spinbox.core.ProductState}]{\sphinxcrossref{ProductState}}}

\end{description}\end{quote}

\end{fulllineitems}

\index{dagger() (spinbox.core.ProductState method)@\spxentry{dagger()}\spxextra{spinbox.core.ProductState method}}

\begin{fulllineitems}
\phantomsection\label{\detokenize{spinbox:spinbox.core.ProductState.dagger}}
\pysigstartsignatures
\pysiglinewithargsret{\sphinxbfcode{\sphinxupquote{dagger}}}{}{{ $\rightarrow$ {\hyperref[\detokenize{spinbox:spinbox.core.ProductState}]{\sphinxcrossref{ProductState}}}}}
\pysigstopsignatures
\sphinxAtStartPar
Hermitian conjugate.
\begin{quote}\begin{description}
\sphinxlineitem{Returns}
\sphinxAtStartPar
The dual \sphinxcode{\sphinxupquote{ProductState}}

\sphinxlineitem{Return type}
\sphinxAtStartPar
{\hyperref[\detokenize{spinbox:spinbox.core.ProductState}]{\sphinxcrossref{ProductState}}}

\end{description}\end{quote}

\end{fulllineitems}

\index{generate\_basis\_states() (spinbox.core.ProductState method)@\spxentry{generate\_basis\_states()}\spxextra{spinbox.core.ProductState method}}

\begin{fulllineitems}
\phantomsection\label{\detokenize{spinbox:spinbox.core.ProductState.generate_basis_states}}
\pysigstartsignatures
\pysiglinewithargsret{\sphinxbfcode{\sphinxupquote{generate\_basis\_states}}}{}{{ $\rightarrow$ list\DUrole{p}{{[}}{\hyperref[\detokenize{spinbox:spinbox.core.ProductState}]{\sphinxcrossref{ProductState}}}\DUrole{p}{{]}}}}
\pysigstopsignatures
\sphinxAtStartPar
Makes a list of corresponding basis vectors.
\begin{quote}\begin{description}
\sphinxlineitem{Returns}
\sphinxAtStartPar
A list of tensor product states that span the Hilbert space.

\sphinxlineitem{Return type}
\sphinxAtStartPar
list{[}{\hyperref[\detokenize{spinbox:spinbox.core.ProductState}]{\sphinxcrossref{ProductState}}}{]}

\end{description}\end{quote}

\end{fulllineitems}

\index{inner() (spinbox.core.ProductState method)@\spxentry{inner()}\spxextra{spinbox.core.ProductState method}}

\begin{fulllineitems}
\phantomsection\label{\detokenize{spinbox:spinbox.core.ProductState.inner}}
\pysigstartsignatures
\pysiglinewithargsret{\sphinxbfcode{\sphinxupquote{inner}}}{\sphinxparam{\DUrole{n}{other}\DUrole{p}{:}\DUrole{w}{ }\DUrole{n}{{\hyperref[\detokenize{spinbox:spinbox.core.ProductState}]{\sphinxcrossref{ProductState}}}}}}{{ $\rightarrow$ complex}}
\pysigstopsignatures
\sphinxAtStartPar
Inner product of two ProductState instances. Orientations must be correct.
\begin{quote}\begin{description}
\sphinxlineitem{Parameters}
\sphinxAtStartPar
\sphinxstyleliteralstrong{\sphinxupquote{other}} ({\hyperref[\detokenize{spinbox:spinbox.core.ProductState}]{\sphinxcrossref{\sphinxstyleliteralemphasis{\sphinxupquote{ProductState}}}}}) \textendash{} The ket of the inner product.

\sphinxlineitem{Returns}
\sphinxAtStartPar
inner product of self (bra) with other (ket)

\sphinxlineitem{Return type}
\sphinxAtStartPar
complex

\end{description}\end{quote}

\end{fulllineitems}

\index{normalize() (spinbox.core.ProductState method)@\spxentry{normalize()}\spxextra{spinbox.core.ProductState method}}

\begin{fulllineitems}
\phantomsection\label{\detokenize{spinbox:spinbox.core.ProductState.normalize}}
\pysigstartsignatures
\pysiglinewithargsret{\sphinxbfcode{\sphinxupquote{normalize}}}{}{{ $\rightarrow$ {\hyperref[\detokenize{spinbox:spinbox.core.ProductState}]{\sphinxcrossref{ProductState}}}}}
\pysigstopsignatures
\sphinxAtStartPar
Normalize so that the inner product of the state with itself is 1.
\begin{quote}\begin{description}
\sphinxlineitem{Returns}
\sphinxAtStartPar
The normalized state.

\sphinxlineitem{Return type}
\sphinxAtStartPar
{\hyperref[\detokenize{spinbox:spinbox.core.ProductState}]{\sphinxcrossref{ProductState}}}

\end{description}\end{quote}

\end{fulllineitems}

\index{outer() (spinbox.core.ProductState method)@\spxentry{outer()}\spxextra{spinbox.core.ProductState method}}

\begin{fulllineitems}
\phantomsection\label{\detokenize{spinbox:spinbox.core.ProductState.outer}}
\pysigstartsignatures
\pysiglinewithargsret{\sphinxbfcode{\sphinxupquote{outer}}}{\sphinxparam{\DUrole{n}{other}\DUrole{p}{:}\DUrole{w}{ }\DUrole{n}{{\hyperref[\detokenize{spinbox:spinbox.core.ProductState}]{\sphinxcrossref{ProductState}}}}}}{{ $\rightarrow$ {\hyperref[\detokenize{spinbox:spinbox.core.ProductState}]{\sphinxcrossref{ProductState}}}}}
\pysigstopsignatures
\sphinxAtStartPar
Outer product of two ProductState instances, producting a ProductOperator instance. Orientations must be correct.
\begin{quote}\begin{description}
\sphinxlineitem{Parameters}
\sphinxAtStartPar
\sphinxstyleliteralstrong{\sphinxupquote{other}} ({\hyperref[\detokenize{spinbox:spinbox.core.ProductState}]{\sphinxcrossref{\sphinxstyleliteralemphasis{\sphinxupquote{ProductState}}}}}) \textendash{} bra part of the outer product

\sphinxlineitem{Returns}
\sphinxAtStartPar
Outer product of self (ket) with other (bra)

\sphinxlineitem{Return type}
\sphinxAtStartPar
{\hyperref[\detokenize{spinbox:spinbox.core.ProductOperator}]{\sphinxcrossref{ProductOperator}}}

\end{description}\end{quote}

\end{fulllineitems}

\index{randomize() (spinbox.core.ProductState method)@\spxentry{randomize()}\spxextra{spinbox.core.ProductState method}}

\begin{fulllineitems}
\phantomsection\label{\detokenize{spinbox:spinbox.core.ProductState.randomize}}
\pysigstartsignatures
\pysiglinewithargsret{\sphinxbfcode{\sphinxupquote{randomize}}}{\sphinxparam{\DUrole{n}{seed}\DUrole{p}{:}\DUrole{w}{ }\DUrole{n}{int}\DUrole{w}{ }\DUrole{o}{=}\DUrole{w}{ }\DUrole{default_value}{None}}}{{ $\rightarrow$ {\hyperref[\detokenize{spinbox:spinbox.core.ProductState}]{\sphinxcrossref{ProductState}}}}}
\pysigstopsignatures
\sphinxAtStartPar
Randomize coefficients.
\begin{quote}\begin{description}
\sphinxlineitem{Parameters}
\sphinxAtStartPar
\sphinxstyleliteralstrong{\sphinxupquote{seed}} (\sphinxstyleliteralemphasis{\sphinxupquote{int}}\sphinxstyleliteralemphasis{\sphinxupquote{, }}\sphinxstyleliteralemphasis{\sphinxupquote{optional}}) \textendash{} RNG seed, defaults to None

\sphinxlineitem{Returns}
\sphinxAtStartPar
A copy of the \sphinxcode{\sphinxupquote{ProductState}} with random complex coefficients, normalized.

\sphinxlineitem{Return type}
\sphinxAtStartPar
{\hyperref[\detokenize{spinbox:spinbox.core.ProductState}]{\sphinxcrossref{ProductState}}}

\end{description}\end{quote}

\end{fulllineitems}

\index{scale\_all() (spinbox.core.ProductState method)@\spxentry{scale\_all()}\spxextra{spinbox.core.ProductState method}}

\begin{fulllineitems}
\phantomsection\label{\detokenize{spinbox:spinbox.core.ProductState.scale_all}}
\pysigstartsignatures
\pysiglinewithargsret{\sphinxbfcode{\sphinxupquote{scale\_all}}}{\sphinxparam{\DUrole{n}{b}\DUrole{p}{:}\DUrole{w}{ }\DUrole{n}{complex}}}{{ $\rightarrow$ {\hyperref[\detokenize{spinbox:spinbox.core.ProductState}]{\sphinxcrossref{ProductState}}}}}
\pysigstopsignatures
\sphinxAtStartPar
Scales an A\sphinxhyphen{}body state by \sphinxcode{\sphinxupquote{b}} by multiplying each one\sphinxhyphen{}body vector by the Ath root of \sphinxcode{\sphinxupquote{b}}.
\begin{quote}\begin{description}
\sphinxlineitem{Parameters}
\sphinxAtStartPar
\sphinxstyleliteralstrong{\sphinxupquote{b}} (\sphinxstyleliteralemphasis{\sphinxupquote{complex}}) \textendash{} scalar

\sphinxlineitem{Returns}
\sphinxAtStartPar
The scaled state

\sphinxlineitem{Return type}
\sphinxAtStartPar
{\hyperref[\detokenize{spinbox:spinbox.core.ProductState}]{\sphinxcrossref{ProductState}}}

\end{description}\end{quote}

\end{fulllineitems}

\index{scale\_one() (spinbox.core.ProductState method)@\spxentry{scale\_one()}\spxextra{spinbox.core.ProductState method}}

\begin{fulllineitems}
\phantomsection\label{\detokenize{spinbox:spinbox.core.ProductState.scale_one}}
\pysigstartsignatures
\pysiglinewithargsret{\sphinxbfcode{\sphinxupquote{scale\_one}}}{\sphinxparam{\DUrole{n}{particle\_index}\DUrole{p}{:}\DUrole{w}{ }\DUrole{n}{int}}\sphinxparamcomma \sphinxparam{\DUrole{n}{b}\DUrole{p}{:}\DUrole{w}{ }\DUrole{n}{complex}}}{{ $\rightarrow$ {\hyperref[\detokenize{spinbox:spinbox.core.ProductState}]{\sphinxcrossref{ProductState}}}}}
\pysigstopsignatures
\sphinxAtStartPar
Multiplies a single particle vector by a number.
\begin{quote}\begin{description}
\sphinxlineitem{Parameters}\begin{itemize}
\item {} 
\sphinxAtStartPar
\sphinxstyleliteralstrong{\sphinxupquote{particle\_index}} (\sphinxstyleliteralemphasis{\sphinxupquote{int}}) \textendash{} Index of particle, starting from 0.

\item {} 
\sphinxAtStartPar
\sphinxstyleliteralstrong{\sphinxupquote{b}} (\sphinxstyleliteralemphasis{\sphinxupquote{complex}}) \textendash{} Scalar

\end{itemize}

\sphinxlineitem{Returns}
\sphinxAtStartPar
A copy of the \sphinxcode{\sphinxupquote{ProductState}} with the one particle scaled.

\sphinxlineitem{Return type}
\sphinxAtStartPar
{\hyperref[\detokenize{spinbox:spinbox.core.ProductState}]{\sphinxcrossref{ProductState}}}

\end{description}\end{quote}

\end{fulllineitems}

\index{to\_list() (spinbox.core.ProductState method)@\spxentry{to\_list()}\spxextra{spinbox.core.ProductState method}}

\begin{fulllineitems}
\phantomsection\label{\detokenize{spinbox:spinbox.core.ProductState.to_list}}
\pysigstartsignatures
\pysiglinewithargsret{\sphinxbfcode{\sphinxupquote{to\_list}}}{}{{ $\rightarrow$ list}}
\pysigstopsignatures\begin{quote}\begin{description}
\sphinxlineitem{Returns}
\sphinxAtStartPar
A list of one\sphinxhyphen{}body vectors

\sphinxlineitem{Return type}
\sphinxAtStartPar
list{[}numpy.ndarray{]}

\end{description}\end{quote}

\end{fulllineitems}

\index{to\_manybody\_basis() (spinbox.core.ProductState method)@\spxentry{to\_manybody\_basis()}\spxextra{spinbox.core.ProductState method}}

\begin{fulllineitems}
\phantomsection\label{\detokenize{spinbox:spinbox.core.ProductState.to_manybody_basis}}
\pysigstartsignatures
\pysiglinewithargsret{\sphinxbfcode{\sphinxupquote{to\_manybody\_basis}}}{}{{ $\rightarrow$ {\hyperref[\detokenize{spinbox:spinbox.core.HilbertState}]{\sphinxcrossref{HilbertState}}}}}
\pysigstopsignatures
\sphinxAtStartPar
Projects to the many\sphinxhyphen{}body basis.
\begin{quote}\begin{description}
\sphinxlineitem{Returns}
\sphinxAtStartPar
The Kronecker product of the \sphinxcode{\sphinxupquote{ProductState}}.

\sphinxlineitem{Return type}
\sphinxAtStartPar
{\hyperref[\detokenize{spinbox:spinbox.core.HilbertState}]{\sphinxcrossref{HilbertState}}}

\end{description}\end{quote}

\end{fulllineitems}

\index{zero() (spinbox.core.ProductState method)@\spxentry{zero()}\spxextra{spinbox.core.ProductState method}}

\begin{fulllineitems}
\phantomsection\label{\detokenize{spinbox:spinbox.core.ProductState.zero}}
\pysigstartsignatures
\pysiglinewithargsret{\sphinxbfcode{\sphinxupquote{zero}}}{}{{ $\rightarrow$ {\hyperref[\detokenize{spinbox:spinbox.core.ProductState}]{\sphinxcrossref{ProductState}}}}}
\pysigstopsignatures
\sphinxAtStartPar
Set all coefficients to zero.
\begin{quote}\begin{description}
\sphinxlineitem{Returns}
\sphinxAtStartPar
A copy of \sphinxcode{\sphinxupquote{ProductState}} with all coefficients set to zero.

\sphinxlineitem{Return type}
\sphinxAtStartPar
{\hyperref[\detokenize{spinbox:spinbox.core.ProductState}]{\sphinxcrossref{ProductState}}}

\end{description}\end{quote}

\end{fulllineitems}


\end{fulllineitems}

\index{Propagator (class in spinbox.core)@\spxentry{Propagator}\spxextra{class in spinbox.core}}

\begin{fulllineitems}
\phantomsection\label{\detokenize{spinbox:spinbox.core.Propagator}}
\pysigstartsignatures
\pysiglinewithargsret{\sphinxbfcode{\sphinxupquote{class\DUrole{w}{ }}}\sphinxcode{\sphinxupquote{spinbox.core.}}\sphinxbfcode{\sphinxupquote{Propagator}}}{\sphinxparam{\DUrole{n}{n\_particles}}\sphinxparamcomma \sphinxparam{\DUrole{n}{dt}\DUrole{p}{:}\DUrole{w}{ }\DUrole{n}{float}}\sphinxparamcomma \sphinxparam{\DUrole{n}{isospin}\DUrole{o}{=}\DUrole{default_value}{True}}\sphinxparamcomma \sphinxparam{\DUrole{n}{include\_prefactors}\DUrole{o}{=}\DUrole{default_value}{True}}}{}
\pysigstopsignatures
\sphinxAtStartPar
Bases: \sphinxcode{\sphinxupquote{object}}

\end{fulllineitems}

\index{SigmaCoupling (class in spinbox.core)@\spxentry{SigmaCoupling}\spxextra{class in spinbox.core}}

\begin{fulllineitems}
\phantomsection\label{\detokenize{spinbox:spinbox.core.SigmaCoupling}}
\pysigstartsignatures
\pysiglinewithargsret{\sphinxbfcode{\sphinxupquote{class\DUrole{w}{ }}}\sphinxcode{\sphinxupquote{spinbox.core.}}\sphinxbfcode{\sphinxupquote{SigmaCoupling}}}{\sphinxparam{\DUrole{n}{n\_particles}}\sphinxparamcomma \sphinxparam{\DUrole{n}{file}\DUrole{o}{=}\DUrole{default_value}{None}}}{}
\pysigstopsignatures
\sphinxAtStartPar
Bases: {\hyperref[\detokenize{spinbox:spinbox.core.Coupling}]{\sphinxcrossref{\sphinxcode{\sphinxupquote{Coupling}}}}}

\sphinxAtStartPar
The coupling matrix \(A^\sigma_{\alpha i \beta j}\)

\sphinxAtStartPar
for i, j = 0 .. n\_particles \sphinxhyphen{} 1
and a, b = 0, 1, 2  (x, y, z)
\index{random() (spinbox.core.SigmaCoupling method)@\spxentry{random()}\spxextra{spinbox.core.SigmaCoupling method}}

\begin{fulllineitems}
\phantomsection\label{\detokenize{spinbox:spinbox.core.SigmaCoupling.random}}
\pysigstartsignatures
\pysiglinewithargsret{\sphinxbfcode{\sphinxupquote{random}}}{\sphinxparam{\DUrole{n}{scale}}\sphinxparamcomma \sphinxparam{\DUrole{n}{seed}\DUrole{o}{=}\DUrole{default_value}{0}}}{}
\pysigstopsignatures
\end{fulllineitems}

\index{validate() (spinbox.core.SigmaCoupling method)@\spxentry{validate()}\spxextra{spinbox.core.SigmaCoupling method}}

\begin{fulllineitems}
\phantomsection\label{\detokenize{spinbox:spinbox.core.SigmaCoupling.validate}}
\pysigstartsignatures
\pysiglinewithargsret{\sphinxbfcode{\sphinxupquote{validate}}}{}{}
\pysigstopsignatures
\end{fulllineitems}


\end{fulllineitems}

\index{SigmaTauCoupling (class in spinbox.core)@\spxentry{SigmaTauCoupling}\spxextra{class in spinbox.core}}

\begin{fulllineitems}
\phantomsection\label{\detokenize{spinbox:spinbox.core.SigmaTauCoupling}}
\pysigstartsignatures
\pysiglinewithargsret{\sphinxbfcode{\sphinxupquote{class\DUrole{w}{ }}}\sphinxcode{\sphinxupquote{spinbox.core.}}\sphinxbfcode{\sphinxupquote{SigmaTauCoupling}}}{\sphinxparam{\DUrole{n}{n\_particles}}\sphinxparamcomma \sphinxparam{\DUrole{n}{file}\DUrole{o}{=}\DUrole{default_value}{None}}}{}
\pysigstopsignatures
\sphinxAtStartPar
Bases: {\hyperref[\detokenize{spinbox:spinbox.core.Coupling}]{\sphinxcrossref{\sphinxcode{\sphinxupquote{Coupling}}}}}

\sphinxAtStartPar
container class for couplings A \textasciicircum{} sigma tau (a,i,b,j)
for i, j = 0 .. n\_particles \sphinxhyphen{} 1
and a, b = 0, 1, 2  (x, y, z)

\sphinxAtStartPar
Note that there are no dimensional indices for tau because the tau factor is a dot product, and thus the couplings are the same over dimensions.
\index{random() (spinbox.core.SigmaTauCoupling method)@\spxentry{random()}\spxextra{spinbox.core.SigmaTauCoupling method}}

\begin{fulllineitems}
\phantomsection\label{\detokenize{spinbox:spinbox.core.SigmaTauCoupling.random}}
\pysigstartsignatures
\pysiglinewithargsret{\sphinxbfcode{\sphinxupquote{random}}}{\sphinxparam{\DUrole{n}{scale}}\sphinxparamcomma \sphinxparam{\DUrole{n}{seed}\DUrole{o}{=}\DUrole{default_value}{0}}}{}
\pysigstopsignatures
\end{fulllineitems}

\index{validate() (spinbox.core.SigmaTauCoupling method)@\spxentry{validate()}\spxextra{spinbox.core.SigmaTauCoupling method}}

\begin{fulllineitems}
\phantomsection\label{\detokenize{spinbox:spinbox.core.SigmaTauCoupling.validate}}
\pysigstartsignatures
\pysiglinewithargsret{\sphinxbfcode{\sphinxupquote{validate}}}{}{}
\pysigstopsignatures
\end{fulllineitems}


\end{fulllineitems}

\index{SpinOrbitCoupling (class in spinbox.core)@\spxentry{SpinOrbitCoupling}\spxextra{class in spinbox.core}}

\begin{fulllineitems}
\phantomsection\label{\detokenize{spinbox:spinbox.core.SpinOrbitCoupling}}
\pysigstartsignatures
\pysiglinewithargsret{\sphinxbfcode{\sphinxupquote{class\DUrole{w}{ }}}\sphinxcode{\sphinxupquote{spinbox.core.}}\sphinxbfcode{\sphinxupquote{SpinOrbitCoupling}}}{\sphinxparam{\DUrole{n}{n\_particles}}\sphinxparamcomma \sphinxparam{\DUrole{n}{file}\DUrole{o}{=}\DUrole{default_value}{None}}}{}
\pysigstopsignatures
\sphinxAtStartPar
Bases: {\hyperref[\detokenize{spinbox:spinbox.core.Coupling}]{\sphinxcrossref{\sphinxcode{\sphinxupquote{Coupling}}}}}

\sphinxAtStartPar
container class for couplings g\_LS (a,i)
for i = 0 .. n\_particles \sphinxhyphen{} 1
and a = 0, 1, 2  (x, y, z)
\index{random() (spinbox.core.SpinOrbitCoupling method)@\spxentry{random()}\spxextra{spinbox.core.SpinOrbitCoupling method}}

\begin{fulllineitems}
\phantomsection\label{\detokenize{spinbox:spinbox.core.SpinOrbitCoupling.random}}
\pysigstartsignatures
\pysiglinewithargsret{\sphinxbfcode{\sphinxupquote{random}}}{\sphinxparam{\DUrole{n}{scale}}\sphinxparamcomma \sphinxparam{\DUrole{n}{seed}\DUrole{o}{=}\DUrole{default_value}{0}}}{}
\pysigstopsignatures
\end{fulllineitems}

\index{validate() (spinbox.core.SpinOrbitCoupling method)@\spxentry{validate()}\spxextra{spinbox.core.SpinOrbitCoupling method}}

\begin{fulllineitems}
\phantomsection\label{\detokenize{spinbox:spinbox.core.SpinOrbitCoupling.validate}}
\pysigstartsignatures
\pysiglinewithargsret{\sphinxbfcode{\sphinxupquote{validate}}}{}{}
\pysigstopsignatures
\end{fulllineitems}


\end{fulllineitems}

\index{TauCoupling (class in spinbox.core)@\spxentry{TauCoupling}\spxextra{class in spinbox.core}}

\begin{fulllineitems}
\phantomsection\label{\detokenize{spinbox:spinbox.core.TauCoupling}}
\pysigstartsignatures
\pysiglinewithargsret{\sphinxbfcode{\sphinxupquote{class\DUrole{w}{ }}}\sphinxcode{\sphinxupquote{spinbox.core.}}\sphinxbfcode{\sphinxupquote{TauCoupling}}}{\sphinxparam{\DUrole{n}{n\_particles}}\sphinxparamcomma \sphinxparam{\DUrole{n}{file}\DUrole{o}{=}\DUrole{default_value}{None}}}{}
\pysigstopsignatures
\sphinxAtStartPar
Bases: {\hyperref[\detokenize{spinbox:spinbox.core.Coupling}]{\sphinxcrossref{\sphinxcode{\sphinxupquote{Coupling}}}}}

\sphinxAtStartPar
container class for couplings A\textasciicircum{}tau (i,j)
for i, j = 0 .. n\_particles \sphinxhyphen{} 1
\index{random() (spinbox.core.TauCoupling method)@\spxentry{random()}\spxextra{spinbox.core.TauCoupling method}}

\begin{fulllineitems}
\phantomsection\label{\detokenize{spinbox:spinbox.core.TauCoupling.random}}
\pysigstartsignatures
\pysiglinewithargsret{\sphinxbfcode{\sphinxupquote{random}}}{\sphinxparam{\DUrole{n}{scale}}\sphinxparamcomma \sphinxparam{\DUrole{n}{seed}\DUrole{o}{=}\DUrole{default_value}{0}}}{}
\pysigstopsignatures
\end{fulllineitems}

\index{validate() (spinbox.core.TauCoupling method)@\spxentry{validate()}\spxextra{spinbox.core.TauCoupling method}}

\begin{fulllineitems}
\phantomsection\label{\detokenize{spinbox:spinbox.core.TauCoupling.validate}}
\pysigstartsignatures
\pysiglinewithargsret{\sphinxbfcode{\sphinxupquote{validate}}}{}{}
\pysigstopsignatures
\end{fulllineitems}


\end{fulllineitems}

\index{ThreeBodyCoupling (class in spinbox.core)@\spxentry{ThreeBodyCoupling}\spxextra{class in spinbox.core}}

\begin{fulllineitems}
\phantomsection\label{\detokenize{spinbox:spinbox.core.ThreeBodyCoupling}}
\pysigstartsignatures
\pysiglinewithargsret{\sphinxbfcode{\sphinxupquote{class\DUrole{w}{ }}}\sphinxcode{\sphinxupquote{spinbox.core.}}\sphinxbfcode{\sphinxupquote{ThreeBodyCoupling}}}{\sphinxparam{\DUrole{n}{n\_particles}}\sphinxparamcomma \sphinxparam{\DUrole{n}{file}\DUrole{o}{=}\DUrole{default_value}{None}}}{}
\pysigstopsignatures
\sphinxAtStartPar
Bases: {\hyperref[\detokenize{spinbox:spinbox.core.Coupling}]{\sphinxcrossref{\sphinxcode{\sphinxupquote{Coupling}}}}}

\sphinxAtStartPar
container class for couplings A(a,i,b,j,c,k)
for i, j, k = 0 .. n\_particles \sphinxhyphen{} 1
and a = 0, 1, 2  (x, y, z)
\index{random() (spinbox.core.ThreeBodyCoupling method)@\spxentry{random()}\spxextra{spinbox.core.ThreeBodyCoupling method}}

\begin{fulllineitems}
\phantomsection\label{\detokenize{spinbox:spinbox.core.ThreeBodyCoupling.random}}
\pysigstartsignatures
\pysiglinewithargsret{\sphinxbfcode{\sphinxupquote{random}}}{\sphinxparam{\DUrole{n}{scale}}\sphinxparamcomma \sphinxparam{\DUrole{n}{seed}\DUrole{o}{=}\DUrole{default_value}{0}}}{}
\pysigstopsignatures
\end{fulllineitems}

\index{validate() (spinbox.core.ThreeBodyCoupling method)@\spxentry{validate()}\spxextra{spinbox.core.ThreeBodyCoupling method}}

\begin{fulllineitems}
\phantomsection\label{\detokenize{spinbox:spinbox.core.ThreeBodyCoupling.validate}}
\pysigstartsignatures
\pysiglinewithargsret{\sphinxbfcode{\sphinxupquote{validate}}}{}{}
\pysigstopsignatures
\end{fulllineitems}


\end{fulllineitems}

\index{carctanh() (in module spinbox.core)@\spxentry{carctanh()}\spxextra{in module spinbox.core}}

\begin{fulllineitems}
\phantomsection\label{\detokenize{spinbox:spinbox.core.carctanh}}
\pysigstartsignatures
\pysiglinewithargsret{\sphinxcode{\sphinxupquote{spinbox.core.}}\sphinxbfcode{\sphinxupquote{carctanh}}}{\sphinxparam{\DUrole{n}{x}}}{}
\pysigstopsignatures
\sphinxAtStartPar
Complex incerse hyp. tangent

\end{fulllineitems}

\index{ccos() (in module spinbox.core)@\spxentry{ccos()}\spxextra{in module spinbox.core}}

\begin{fulllineitems}
\phantomsection\label{\detokenize{spinbox:spinbox.core.ccos}}
\pysigstartsignatures
\pysiglinewithargsret{\sphinxcode{\sphinxupquote{spinbox.core.}}\sphinxbfcode{\sphinxupquote{ccos}}}{\sphinxparam{\DUrole{n}{x}}}{}
\pysigstopsignatures
\sphinxAtStartPar
Complex cosine

\end{fulllineitems}

\index{ccosh() (in module spinbox.core)@\spxentry{ccosh()}\spxextra{in module spinbox.core}}

\begin{fulllineitems}
\phantomsection\label{\detokenize{spinbox:spinbox.core.ccosh}}
\pysigstartsignatures
\pysiglinewithargsret{\sphinxcode{\sphinxupquote{spinbox.core.}}\sphinxbfcode{\sphinxupquote{ccosh}}}{\sphinxparam{\DUrole{n}{x}}}{}
\pysigstopsignatures
\sphinxAtStartPar
Complex hyp. cosine

\end{fulllineitems}

\index{cexp() (in module spinbox.core)@\spxentry{cexp()}\spxextra{in module spinbox.core}}

\begin{fulllineitems}
\phantomsection\label{\detokenize{spinbox:spinbox.core.cexp}}
\pysigstartsignatures
\pysiglinewithargsret{\sphinxcode{\sphinxupquote{spinbox.core.}}\sphinxbfcode{\sphinxupquote{cexp}}}{\sphinxparam{\DUrole{n}{x}}}{}
\pysigstopsignatures
\sphinxAtStartPar
Complex exponential

\end{fulllineitems}

\index{csin() (in module spinbox.core)@\spxentry{csin()}\spxextra{in module spinbox.core}}

\begin{fulllineitems}
\phantomsection\label{\detokenize{spinbox:spinbox.core.csin}}
\pysigstartsignatures
\pysiglinewithargsret{\sphinxcode{\sphinxupquote{spinbox.core.}}\sphinxbfcode{\sphinxupquote{csin}}}{\sphinxparam{\DUrole{n}{x}}}{}
\pysigstopsignatures
\sphinxAtStartPar
Complex sine

\end{fulllineitems}

\index{csinh() (in module spinbox.core)@\spxentry{csinh()}\spxextra{in module spinbox.core}}

\begin{fulllineitems}
\phantomsection\label{\detokenize{spinbox:spinbox.core.csinh}}
\pysigstartsignatures
\pysiglinewithargsret{\sphinxcode{\sphinxupquote{spinbox.core.}}\sphinxbfcode{\sphinxupquote{csinh}}}{\sphinxparam{\DUrole{n}{x}}}{}
\pysigstopsignatures
\sphinxAtStartPar
Complex hyp. sine

\end{fulllineitems}

\index{csqrt() (in module spinbox.core)@\spxentry{csqrt()}\spxextra{in module spinbox.core}}

\begin{fulllineitems}
\phantomsection\label{\detokenize{spinbox:spinbox.core.csqrt}}
\pysigstartsignatures
\pysiglinewithargsret{\sphinxcode{\sphinxupquote{spinbox.core.}}\sphinxbfcode{\sphinxupquote{csqrt}}}{\sphinxparam{\DUrole{n}{x}}}{}
\pysigstopsignatures
\sphinxAtStartPar
Complex square root

\end{fulllineitems}

\index{ctanh() (in module spinbox.core)@\spxentry{ctanh()}\spxextra{in module spinbox.core}}

\begin{fulllineitems}
\phantomsection\label{\detokenize{spinbox:spinbox.core.ctanh}}
\pysigstartsignatures
\pysiglinewithargsret{\sphinxcode{\sphinxupquote{spinbox.core.}}\sphinxbfcode{\sphinxupquote{ctanh}}}{\sphinxparam{\DUrole{n}{x}}}{}
\pysigstopsignatures
\sphinxAtStartPar
Complex hyp. tangent

\end{fulllineitems}

\index{interaction\_indices() (in module spinbox.core)@\spxentry{interaction\_indices()}\spxextra{in module spinbox.core}}

\begin{fulllineitems}
\phantomsection\label{\detokenize{spinbox:spinbox.core.interaction_indices}}
\pysigstartsignatures
\pysiglinewithargsret{\sphinxcode{\sphinxupquote{spinbox.core.}}\sphinxbfcode{\sphinxupquote{interaction\_indices}}}{\sphinxparam{\DUrole{n}{n}\DUrole{p}{:}\DUrole{w}{ }\DUrole{n}{int}}\sphinxparamcomma \sphinxparam{\DUrole{n}{m}\DUrole{o}{=}\DUrole{default_value}{2}}}{{ $\rightarrow$ list}}
\pysigstopsignatures
\sphinxAtStartPar
returns a list of all possible m\sphinxhyphen{}plets of n objects (labelled 0 to n\sphinxhyphen{}1)
default: m=2, giving all possible pairs
for m=1, returns a range(0, n\sphinxhyphen{}1)
:param n: number of items
:type n: int
:param m: size of tuplet, defaults to 2
:type m: int, optional
:return: list of possible m\sphinxhyphen{}plets of n items
:rtype: list

\end{fulllineitems}

\index{pauli() (in module spinbox.core)@\spxentry{pauli()}\spxextra{in module spinbox.core}}

\begin{fulllineitems}
\phantomsection\label{\detokenize{spinbox:spinbox.core.pauli}}
\pysigstartsignatures
\pysiglinewithargsret{\sphinxcode{\sphinxupquote{spinbox.core.}}\sphinxbfcode{\sphinxupquote{pauli}}}{\sphinxparam{\DUrole{n}{arg}}}{{ $\rightarrow$ ndarray}}
\pysigstopsignatures
\sphinxAtStartPar
Pauli matrix x, y, z, or a list of all three
\begin{quote}\begin{description}
\sphinxlineitem{Parameters}
\sphinxAtStartPar
\sphinxstyleliteralstrong{\sphinxupquote{arg}} (\sphinxstyleliteralemphasis{\sphinxupquote{int}}\sphinxstyleliteralemphasis{\sphinxupquote{ or }}\sphinxstyleliteralemphasis{\sphinxupquote{str}}) \textendash{} 0 or ‘x’ for Pauli x, 1 or ‘y’ for Pauli y, 2 or ‘z’ for Pauli z, ‘list’ for a list of x, y ,z

\sphinxlineitem{Raises}
\sphinxAtStartPar
\sphinxstyleliteralstrong{\sphinxupquote{ValueError}} \textendash{} option not found

\sphinxlineitem{Returns}
\sphinxAtStartPar
Pauli matrix or list

\sphinxlineitem{Return type}
\sphinxAtStartPar
np.ndarray

\end{description}\end{quote}

\end{fulllineitems}

\index{read\_from\_file() (in module spinbox.core)@\spxentry{read\_from\_file()}\spxextra{in module spinbox.core}}

\begin{fulllineitems}
\phantomsection\label{\detokenize{spinbox:spinbox.core.read_from_file}}
\pysigstartsignatures
\pysiglinewithargsret{\sphinxcode{\sphinxupquote{spinbox.core.}}\sphinxbfcode{\sphinxupquote{read\_from\_file}}}{\sphinxparam{\DUrole{n}{filename}\DUrole{p}{:}\DUrole{w}{ }\DUrole{n}{str}}\sphinxparamcomma \sphinxparam{\DUrole{n}{complex}\DUrole{o}{=}\DUrole{default_value}{False}}\sphinxparamcomma \sphinxparam{\DUrole{n}{shape}\DUrole{o}{=}\DUrole{default_value}{None}}\sphinxparamcomma \sphinxparam{\DUrole{n}{order}\DUrole{o}{=}\DUrole{default_value}{\textquotesingle{}F\textquotesingle{}}}}{{ $\rightarrow$ ndarray}}
\pysigstopsignatures
\sphinxAtStartPar
Read numbers from a text file
\begin{quote}\begin{description}
\sphinxlineitem{Parameters}\begin{itemize}
\item {} 
\sphinxAtStartPar
\sphinxstyleliteralstrong{\sphinxupquote{filename}} (\sphinxstyleliteralemphasis{\sphinxupquote{str}}) \textendash{} input file name

\item {} 
\sphinxAtStartPar
\sphinxstyleliteralstrong{\sphinxupquote{complex}} (\sphinxstyleliteralemphasis{\sphinxupquote{bool}}\sphinxstyleliteralemphasis{\sphinxupquote{, }}\sphinxstyleliteralemphasis{\sphinxupquote{optional}}) \textendash{} complex entries, defaults to False

\item {} 
\sphinxAtStartPar
\sphinxstyleliteralstrong{\sphinxupquote{shape}} (\sphinxstyleliteralemphasis{\sphinxupquote{tuple}}\sphinxstyleliteralemphasis{\sphinxupquote{, }}\sphinxstyleliteralemphasis{\sphinxupquote{optional}}) \textendash{} shape of output array, defaults to None

\item {} 
\sphinxAtStartPar
\sphinxstyleliteralstrong{\sphinxupquote{order}} (\sphinxstyleliteralemphasis{\sphinxupquote{str}}\sphinxstyleliteralemphasis{\sphinxupquote{, }}\sphinxstyleliteralemphasis{\sphinxupquote{optional}}) \textendash{} ‘F’ for columns first, otherwise use ‘C’, defaults to ‘F’

\end{itemize}

\sphinxlineitem{Returns}
\sphinxAtStartPar
Numpy array

\sphinxlineitem{Return type}
\sphinxAtStartPar
numpy.ndarray

\end{description}\end{quote}

\end{fulllineitems}

\index{repeated\_kronecker\_product() (in module spinbox.core)@\spxentry{repeated\_kronecker\_product()}\spxextra{in module spinbox.core}}

\begin{fulllineitems}
\phantomsection\label{\detokenize{spinbox:spinbox.core.repeated_kronecker_product}}
\pysigstartsignatures
\pysiglinewithargsret{\sphinxcode{\sphinxupquote{spinbox.core.}}\sphinxbfcode{\sphinxupquote{repeated\_kronecker\_product}}}{\sphinxparam{\DUrole{n}{matrices}\DUrole{p}{:}\DUrole{w}{ }\DUrole{n}{list}}}{{ $\rightarrow$ ndarray}}
\pysigstopsignatures
\sphinxAtStartPar
returns the tensor/kronecker product of a list of arrays
:param matrices: list of matrix factors
:type matrices: list
:return: Kronecker product of input list
“rtype: np.ndarray

\end{fulllineitems}



\subsection{spinbox.extras module}
\label{\detokenize{spinbox:module-spinbox.extras}}\label{\detokenize{spinbox:spinbox-extras-module}}\index{module@\spxentry{module}!spinbox.extras@\spxentry{spinbox.extras}}\index{spinbox.extras@\spxentry{spinbox.extras}!module@\spxentry{module}}\index{chistogram() (in module spinbox.extras)@\spxentry{chistogram()}\spxextra{in module spinbox.extras}}

\begin{fulllineitems}
\phantomsection\label{\detokenize{spinbox:spinbox.extras.chistogram}}
\pysigstartsignatures
\pysiglinewithargsret{\sphinxcode{\sphinxupquote{spinbox.extras.}}\sphinxbfcode{\sphinxupquote{chistogram}}}{\sphinxparam{\DUrole{n}{X}}\sphinxparamcomma \sphinxparam{\DUrole{n}{filename}}\sphinxparamcomma \sphinxparam{\DUrole{n}{title}}\sphinxparamcomma \sphinxparam{\DUrole{n}{bins}\DUrole{o}{=}\DUrole{default_value}{\textquotesingle{}fd\textquotesingle{}}}\sphinxparamcomma \sphinxparam{\DUrole{n}{range}\DUrole{o}{=}\DUrole{default_value}{None}}}{}
\pysigstopsignatures
\sphinxAtStartPar
Complex histogram
\begin{quote}\begin{description}
\sphinxlineitem{Parameters}\begin{itemize}
\item {} 
\sphinxAtStartPar
\sphinxstyleliteralstrong{\sphinxupquote{X}} (\sphinxstyleliteralemphasis{\sphinxupquote{iterable}}) \textendash{} Set of complex numbers

\item {} 
\sphinxAtStartPar
\sphinxstyleliteralstrong{\sphinxupquote{filename}} (\sphinxstyleliteralemphasis{\sphinxupquote{str}}) \textendash{} filename of plot, including suffix (.pdf)

\item {} 
\sphinxAtStartPar
\sphinxstyleliteralstrong{\sphinxupquote{title}} (\sphinxstyleliteralemphasis{\sphinxupquote{str}}) \textendash{} Plot title

\item {} 
\sphinxAtStartPar
\sphinxstyleliteralstrong{\sphinxupquote{bins}} (\sphinxstyleliteralemphasis{\sphinxupquote{str}}\sphinxstyleliteralemphasis{\sphinxupquote{, }}\sphinxstyleliteralemphasis{\sphinxupquote{optional}}) \textendash{} binning algorithm (see matplotlib.pyplot.hist), defaults to ‘fd’ (Freedman\sphinxhyphen{}Diaconis)

\item {} 
\sphinxAtStartPar
\sphinxstyleliteralstrong{\sphinxupquote{range}} (\sphinxstyleliteralemphasis{\sphinxupquote{tuple}}\sphinxstyleliteralemphasis{\sphinxupquote{, }}\sphinxstyleliteralemphasis{\sphinxupquote{optional}}) \textendash{} fixed range to plot, defaults to None

\end{itemize}

\end{description}\end{quote}

\end{fulllineitems}

\index{pmat() (in module spinbox.extras)@\spxentry{pmat()}\spxextra{in module spinbox.extras}}

\begin{fulllineitems}
\phantomsection\label{\detokenize{spinbox:spinbox.extras.pmat}}
\pysigstartsignatures
\pysiglinewithargsret{\sphinxcode{\sphinxupquote{spinbox.extras.}}\sphinxbfcode{\sphinxupquote{pmat}}}{\sphinxparam{\DUrole{n}{x}}\sphinxparamcomma \sphinxparam{\DUrole{n}{heatmap}\DUrole{o}{=}\DUrole{default_value}{False}}\sphinxparamcomma \sphinxparam{\DUrole{n}{lims}\DUrole{o}{=}\DUrole{default_value}{None}}\sphinxparamcomma \sphinxparam{\DUrole{n}{print\_zeros}\DUrole{o}{=}\DUrole{default_value}{False}}}{}
\pysigstopsignatures
\sphinxAtStartPar
Print or plot a complex\sphinxhyphen{}valued matrix
\begin{quote}\begin{description}
\sphinxlineitem{Parameters}\begin{itemize}
\item {} 
\sphinxAtStartPar
\sphinxstyleliteralstrong{\sphinxupquote{x}} (\sphinxstyleliteralemphasis{\sphinxupquote{numpy.ndarray}}) \textendash{} matrix to be plotted

\item {} 
\sphinxAtStartPar
\sphinxstyleliteralstrong{\sphinxupquote{heatmap}} (\sphinxstyleliteralemphasis{\sphinxupquote{bool}}\sphinxstyleliteralemphasis{\sphinxupquote{, }}\sphinxstyleliteralemphasis{\sphinxupquote{optional}}) \textendash{} True if plotting a heatmap, defaults to False

\item {} 
\sphinxAtStartPar
\sphinxstyleliteralstrong{\sphinxupquote{lims}} (\sphinxstyleliteralemphasis{\sphinxupquote{tuple}}\sphinxstyleliteralemphasis{\sphinxupquote{, }}\sphinxstyleliteralemphasis{\sphinxupquote{optional}}) \textendash{} if heatmap, limits for colorbar, defaults to None

\item {} 
\sphinxAtStartPar
\sphinxstyleliteralstrong{\sphinxupquote{print\_zeros}} (\sphinxstyleliteralemphasis{\sphinxupquote{bool}}\sphinxstyleliteralemphasis{\sphinxupquote{, }}\sphinxstyleliteralemphasis{\sphinxupquote{optional}}) \textendash{} True if printing a part if it is all zeros, defaults to False

\end{itemize}

\end{description}\end{quote}

\end{fulllineitems}

\index{sigma\_tau\_matrices\_product() (in module spinbox.extras)@\spxentry{sigma\_tau\_matrices\_product()}\spxextra{in module spinbox.extras}}

\begin{fulllineitems}
\phantomsection\label{\detokenize{spinbox:spinbox.extras.sigma_tau_matrices_product}}
\pysigstartsignatures
\pysiglinewithargsret{\sphinxcode{\sphinxupquote{spinbox.extras.}}\sphinxbfcode{\sphinxupquote{sigma\_tau\_matrices\_product}}}{\sphinxparam{\DUrole{n}{n\_particles}}}{}
\pysigstopsignatures
\sphinxAtStartPar
Pauli sigma and tau matrices in product basis
\begin{quote}\begin{description}
\sphinxlineitem{Parameters}
\sphinxAtStartPar
\sphinxstyleliteralstrong{\sphinxupquote{n\_particles}} (\sphinxstyleliteralemphasis{\sphinxupquote{int}}) \textendash{} number of particles

\sphinxlineitem{Returns}
\sphinxAtStartPar
(sigma matrices, tau matrices)

\sphinxlineitem{Return type}
\sphinxAtStartPar
tuple of lists

\end{description}\end{quote}

\sphinxAtStartPar
usage: 
sigma, tau = sigma\_tau\_matrices\_product(n\_particles) 
sigma{[}dimension\_index{]}

\sphinxAtStartPar
note that I am not using the ProductOperator class here. This is done for memory efficiency.
In the case of Hilbert space calculations, it makes sense to compute the operator matrices beforehand and store them.
In the tensor\sphinxhyphen{}product basis, this would result in most of our memory being taken up by identity matrices.

\end{fulllineitems}

\index{sigma\_tau\_operators\_hilbert() (in module spinbox.extras)@\spxentry{sigma\_tau\_operators\_hilbert()}\spxextra{in module spinbox.extras}}

\begin{fulllineitems}
\phantomsection\label{\detokenize{spinbox:spinbox.extras.sigma_tau_operators_hilbert}}
\pysigstartsignatures
\pysiglinewithargsret{\sphinxcode{\sphinxupquote{spinbox.extras.}}\sphinxbfcode{\sphinxupquote{sigma\_tau\_operators\_hilbert}}}{\sphinxparam{\DUrole{n}{n\_particles}}}{}
\pysigstopsignatures
\sphinxAtStartPar
Pauli sigma and tau operators in Hilbert space
\begin{quote}\begin{description}
\sphinxlineitem{Parameters}
\sphinxAtStartPar
\sphinxstyleliteralstrong{\sphinxupquote{n\_particles}} (\sphinxstyleliteralemphasis{\sphinxupquote{int}}) \textendash{} number of particles

\sphinxlineitem{Returns}
\sphinxAtStartPar
(sigma operators, tau operators)

\sphinxlineitem{Return type}
\sphinxAtStartPar
tuple of lists of lists

\end{description}\end{quote}

\sphinxAtStartPar
usage: 
sigma, tau = sigma\_tau\_operators\_hilbert(n\_particles) 
sigma{[}particle\_index{]}{[}dimension\_index{]}

\end{fulllineitems}

\index{spinor2() (in module spinbox.extras)@\spxentry{spinor2()}\spxextra{in module spinbox.extras}}

\begin{fulllineitems}
\phantomsection\label{\detokenize{spinbox:spinbox.extras.spinor2}}
\pysigstartsignatures
\pysiglinewithargsret{\sphinxcode{\sphinxupquote{spinbox.extras.}}\sphinxbfcode{\sphinxupquote{spinor2}}}{\sphinxparam{\DUrole{n}{state}\DUrole{o}{=}\DUrole{default_value}{\textquotesingle{}up\textquotesingle{}}}\sphinxparamcomma \sphinxparam{\DUrole{n}{ketwise}\DUrole{o}{=}\DUrole{default_value}{True}}\sphinxparamcomma \sphinxparam{\DUrole{n}{seed}\DUrole{o}{=}\DUrole{default_value}{None}}}{}
\pysigstopsignatures
\sphinxAtStartPar
Convenience function for making 2\sphinxhyphen{}dimensional spin state vectors
\begin{quote}\begin{description}
\sphinxlineitem{Parameters}\begin{itemize}
\item {} 
\sphinxAtStartPar
\sphinxstyleliteralstrong{\sphinxupquote{state}} (\sphinxstyleliteralemphasis{\sphinxupquote{str}}\sphinxstyleliteralemphasis{\sphinxupquote{, }}\sphinxstyleliteralemphasis{\sphinxupquote{optional}}) \textendash{} can be one of {[}‘up’, ‘down’, ‘random’, ‘max’{]}, defaults to ‘up’.

\item {} 
\sphinxAtStartPar
\sphinxstyleliteralstrong{\sphinxupquote{ketwise}} (\sphinxstyleliteralemphasis{\sphinxupquote{bool}}\sphinxstyleliteralemphasis{\sphinxupquote{, }}\sphinxstyleliteralemphasis{\sphinxupquote{optional}}) \textendash{} True for column vector, False for row vector, defaults to True

\item {} 
\sphinxAtStartPar
\sphinxstyleliteralstrong{\sphinxupquote{seed}} (\sphinxstyleliteralemphasis{\sphinxupquote{int}}\sphinxstyleliteralemphasis{\sphinxupquote{, }}\sphinxstyleliteralemphasis{\sphinxupquote{optional}}) \textendash{} rng seed, defaults to None

\end{itemize}

\sphinxlineitem{Returns}
\sphinxAtStartPar
your vector

\sphinxlineitem{Return type}
\sphinxAtStartPar
numpy.ndarray

\end{description}\end{quote}

\end{fulllineitems}

\index{spinor4() (in module spinbox.extras)@\spxentry{spinor4()}\spxextra{in module spinbox.extras}}

\begin{fulllineitems}
\phantomsection\label{\detokenize{spinbox:spinbox.extras.spinor4}}
\pysigstartsignatures
\pysiglinewithargsret{\sphinxcode{\sphinxupquote{spinbox.extras.}}\sphinxbfcode{\sphinxupquote{spinor4}}}{\sphinxparam{\DUrole{n}{state}\DUrole{o}{=}\DUrole{default_value}{\textquotesingle{}up\textquotesingle{}}}\sphinxparamcomma \sphinxparam{\DUrole{n}{ketwise}\DUrole{o}{=}\DUrole{default_value}{True}}\sphinxparamcomma \sphinxparam{\DUrole{n}{seed}\DUrole{o}{=}\DUrole{default_value}{None}}}{}
\pysigstopsignatures
\sphinxAtStartPar
Convenience function for making 4\sphinxhyphen{}dimensional spin\sphinxhyphen{}isospin state vectors
\begin{quote}\begin{description}
\sphinxlineitem{Parameters}\begin{itemize}
\item {} 
\sphinxAtStartPar
\sphinxstyleliteralstrong{\sphinxupquote{state}} (\sphinxstyleliteralemphasis{\sphinxupquote{str}}\sphinxstyleliteralemphasis{\sphinxupquote{, }}\sphinxstyleliteralemphasis{\sphinxupquote{optional}}) \textendash{} can be one of {[}‘up’, ‘down’, ‘random’, ‘max’{]}, defaults to ‘up’.

\item {} 
\sphinxAtStartPar
\sphinxstyleliteralstrong{\sphinxupquote{ketwise}} (\sphinxstyleliteralemphasis{\sphinxupquote{bool}}\sphinxstyleliteralemphasis{\sphinxupquote{, }}\sphinxstyleliteralemphasis{\sphinxupquote{optional}}) \textendash{} True for column vector, False for row vector, defaults to True

\item {} 
\sphinxAtStartPar
\sphinxstyleliteralstrong{\sphinxupquote{seed}} (\sphinxstyleliteralemphasis{\sphinxupquote{int}}\sphinxstyleliteralemphasis{\sphinxupquote{, }}\sphinxstyleliteralemphasis{\sphinxupquote{optional}}) \textendash{} rng seed, defaults to None

\end{itemize}

\sphinxlineitem{Returns}
\sphinxAtStartPar
your vector

\sphinxlineitem{Return type}
\sphinxAtStartPar
numpy.ndarray

\end{description}\end{quote}

\end{fulllineitems}



\subsection{Module contents}
\label{\detokenize{spinbox:module-spinbox}}\label{\detokenize{spinbox:module-contents}}\index{module@\spxentry{module}!spinbox@\spxentry{spinbox}}\index{spinbox@\spxentry{spinbox}!module@\spxentry{module}}

\chapter{Indices and tables}
\label{\detokenize{index:indices-and-tables}}\begin{itemize}
\item {} 
\sphinxAtStartPar
\DUrole{xref,std,std-ref}{genindex}

\item {} 
\sphinxAtStartPar
\DUrole{xref,std,std-ref}{modindex}

\item {} 
\sphinxAtStartPar
\DUrole{xref,std,std-ref}{search}

\end{itemize}


\renewcommand{\indexname}{Python Module Index}
\begin{sphinxtheindex}
\let\bigletter\sphinxstyleindexlettergroup
\bigletter{s}
\item\relax\sphinxstyleindexentry{spinbox}\sphinxstyleindexpageref{spinbox:\detokenize{module-spinbox}}
\item\relax\sphinxstyleindexentry{spinbox.core}\sphinxstyleindexpageref{spinbox:\detokenize{module-spinbox.core}}
\item\relax\sphinxstyleindexentry{spinbox.extras}\sphinxstyleindexpageref{spinbox:\detokenize{module-spinbox.extras}}
\end{sphinxtheindex}

\renewcommand{\indexname}{Index}
\printindex
\end{document}